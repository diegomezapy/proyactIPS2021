\subsubsection{Stock de Jubilaciones y Pensiones}

A diciembre del 2020, el IPS contaba con xx beneficios otorgados, de los
cuales el xx\% correspondían a Jubilación por Invalidez, el xx\% a
Jubilación Ordinaria, el xx\% a Jubilación Proporcional y el xx\% a
Pensiones (Ver Tabla xx).

El Gráfico xx ilustra la evolución del stock
\footnote{El término "stock" hace referencia a la cantidad total de beneficios pagados en J y P al último día de cada mes.  Los beneficios no incluyen el Complemento al SML ni los pagos por única vez (subsidi
os).} en J y P, en el cual se observa una tendencia creciente en los
beneficios otorgados en Jubilaciones Ordinarias y Jubilaciones
Proporcionales a lo largo de los últimos 5 años. En cuanto a las
Pensiones Derechohabiente y las Jubilaciones por Inval idez, la
tendencia revela estabilidad, es decir, se mantuvieron prácticamente
constantes a lo largo del intervalo estudiado.

Se aprecia en el Gráfico 3.14 que los montos nominales promedios del
stock de J y P, han tenido un incremento sostenido en los diferentes
tipos de prestaciones durante los últimos 5 años. Este aumento es
especialmente notorio en los pagos otorgados por Jubilaciones
Ordinarias, en donde se observa una suba de Gs. xxxxxxx entre los años
xx y xx, equivalente a un incremento del xx \%; seguido de los montos
por Jubilación Proporcional con un aumento del xx\%.

Por otra parte, las Jubilaciones por Invalidez y las Pensiones
Derechohabiente han tenido un incremento del xx\% y xx\%
respectivamente. En tanto que los Complementos al SML se han mantenido
sin grandes variaciones en el transcurso de dichos años.

\begin{table}[H]
\begin{center}
\caption{\bf{Evolución de la cantidad total beneficios pagados en Jubilaciones y Pensiones en el año 2020 (Stock).}}
\begin{tabular}{l|rrrrrrrrrrrrrrr}
\scriptsize
%\begin{tabular}{lllllllllllll}
\cline{1-13}
\multicolumn{1}{c}{} &
  \multicolumn{12}{|c}{Mes de pago} \\
\multicolumn{1}{c}{} &
  \multicolumn{1}{|r}{1} &
  \multicolumn{1}{r}{2} &
  \multicolumn{1}{r}{3} &
  \multicolumn{1}{r}{4} &
  \multicolumn{1}{r}{5} &
  \multicolumn{1}{r}{6} &
  \multicolumn{1}{r}{7} &
  \multicolumn{1}{r}{8} &
  \multicolumn{1}{r}{9} &
  \multicolumn{1}{r}{10} &
  \multicolumn{1}{r}{11} &
  \multicolumn{1}{r}{12} \\
\cline{1-13}
\multicolumn{1}{l}{Clasificación} &
  \multicolumn{1}{|r}{} &
  \multicolumn{1}{r}{} &
  \multicolumn{1}{r}{} &
  \multicolumn{1}{r}{} &
  \multicolumn{1}{r}{} &
  \multicolumn{1}{r}{} &
  \multicolumn{1}{r}{} &
  \multicolumn{1}{r}{} &
  \multicolumn{1}{r}{} &
  \multicolumn{1}{r}{} &
  \multicolumn{1}{r}{} &
  \multicolumn{1}{r}{} \\
\multicolumn{1}{l}{\hspace{1em}CSML} &
  \multicolumn{1}{|r}{7.890} &
  \multicolumn{1}{r}{12.139} &
  \multicolumn{1}{r}{12.253} &
  \multicolumn{1}{r}{12.172} &
  \multicolumn{1}{r}{12.186} &
  \multicolumn{1}{r}{12.289} &
  \multicolumn{1}{r}{12.596} &
  \multicolumn{1}{r}{12.677} &
  \multicolumn{1}{r}{12.698} &
  \multicolumn{1}{r}{12.725} &
  \multicolumn{1}{r}{12.718} &
  \multicolumn{1}{r}{12.754} \\
\multicolumn{1}{l}{\hspace{1em}JIP} &
  \multicolumn{1}{|r}{3.979} &
  \multicolumn{1}{r}{3.971} &
  \multicolumn{1}{r}{3.987} &
  \multicolumn{1}{r}{3.972} &
  \multicolumn{1}{r}{3.981} &
  \multicolumn{1}{r}{3.954} &
  \multicolumn{1}{r}{3.950} &
  \multicolumn{1}{r}{3.955} &
  \multicolumn{1}{r}{3.956} &
  \multicolumn{1}{r}{3.956} &
  \multicolumn{1}{r}{3.939} &
  \multicolumn{1}{r}{3.932} \\
\multicolumn{1}{l}{\hspace{1em}JIT} &
  \multicolumn{1}{|r}{339} &
  \multicolumn{1}{r}{334} &
  \multicolumn{1}{r}{331} &
  \multicolumn{1}{r}{298} &
  \multicolumn{1}{r}{296} &
  \multicolumn{1}{r}{289} &
  \multicolumn{1}{r}{293} &
  \multicolumn{1}{r}{281} &
  \multicolumn{1}{r}{284} &
  \multicolumn{1}{r}{281} &
  \multicolumn{1}{r}{258} &
  \multicolumn{1}{r}{261} \\
\multicolumn{1}{l}{\hspace{1em}JO} &
  \multicolumn{1}{|r}{37.531} &
  \multicolumn{1}{r}{37.510} &
  \multicolumn{1}{r}{37.761} &
  \multicolumn{1}{r}{37.726} &
  \multicolumn{1}{r}{37.968} &
  \multicolumn{1}{r}{38.104} &
  \multicolumn{1}{r}{38.329} &
  \multicolumn{1}{r}{38.436} &
  \multicolumn{1}{r}{38.530} &
  \multicolumn{1}{r}{38.705} &
  \multicolumn{1}{r}{38.704} &
  \multicolumn{1}{r}{38.790} \\
\multicolumn{1}{l}{\hspace{1em}JP} &
  \multicolumn{1}{|r}{10.112} &
  \multicolumn{1}{r}{10.168} &
  \multicolumn{1}{r}{10.265} &
  \multicolumn{1}{r}{10.278} &
  \multicolumn{1}{r}{10.375} &
  \multicolumn{1}{r}{10.472} &
  \multicolumn{1}{r}{10.592} &
  \multicolumn{1}{r}{10.685} &
  \multicolumn{1}{r}{10.780} &
  \multicolumn{1}{r}{10.886} &
  \multicolumn{1}{r}{10.922} &
  \multicolumn{1}{r}{10.986} \\
\multicolumn{1}{l}{\hspace{1em}BAA} &
  \multicolumn{1}{|r}{240} &
  \multicolumn{1}{r}{16} &
  \multicolumn{1}{r}{13} &
  \multicolumn{1}{r}{5} &
  \multicolumn{1}{r}{} &
  \multicolumn{1}{r}{} &
  \multicolumn{1}{r}{23} &
  \multicolumn{1}{r}{1} &
  \multicolumn{1}{r}{2} &
  \multicolumn{1}{r}{4} &
  \multicolumn{1}{r}{63.887} &
  \multicolumn{1}{r}{581} \\
\multicolumn{1}{l}{\hspace{1em}PDH} &
  \multicolumn{1}{|r}{13.663} &
  \multicolumn{1}{r}{13.627} &
  \multicolumn{1}{r}{13.646} &
  \multicolumn{1}{r}{13.576} &
  \multicolumn{1}{r}{13.556} &
  \multicolumn{1}{r}{13.563} &
  \multicolumn{1}{r}{13.688} &
  \multicolumn{1}{r}{13.828} &
  \multicolumn{1}{r}{13.878} &
  \multicolumn{1}{r}{13.961} &
  \multicolumn{1}{r}{13.967} &
  \multicolumn{1}{r}{14.076} \\
\multicolumn{1}{l}{\hspace{1em}JCON} &
  \multicolumn{1}{|r}{108} &
  \multicolumn{1}{r}{384} &
  \multicolumn{1}{r}{388} &
  \multicolumn{1}{r}{386} &
  \multicolumn{1}{r}{388} &
  \multicolumn{1}{r}{389} &
  \multicolumn{1}{r}{398} &
  \multicolumn{1}{r}{397} &
  \multicolumn{1}{r}{390} &
  \multicolumn{1}{r}{384} &
  \multicolumn{1}{r}{389} &
  \multicolumn{1}{r}{387} \\
\multicolumn{1}{l}{\hspace{1em}JINT} &
  \multicolumn{1}{|r}{} &
  \multicolumn{1}{r}{} &
  \multicolumn{1}{r}{12} &
  \multicolumn{1}{r}{12} &
  \multicolumn{1}{r}{12} &
  \multicolumn{1}{r}{13} &
  \multicolumn{1}{r}{13} &
  \multicolumn{1}{r}{13} &
  \multicolumn{1}{r}{13} &
  \multicolumn{1}{r}{13} &
  \multicolumn{1}{r}{15} &
  \multicolumn{1}{r}{14} \\
\multicolumn{1}{l}{\hspace{1em}LQVNP} &
  \multicolumn{1}{|r}{48} &
  \multicolumn{1}{r}{9} &
  \multicolumn{1}{r}{15} &
  \multicolumn{1}{r}{7} &
  \multicolumn{1}{r}{7} &
  \multicolumn{1}{r}{11} &
  \multicolumn{1}{r}{42} &
  \multicolumn{1}{r}{33} &
  \multicolumn{1}{r}{21} &
  \multicolumn{1}{r}{30} &
  \multicolumn{1}{r}{36} &
  \multicolumn{1}{r}{25} \\
\multicolumn{1}{l}{\hspace{1em}GF} &
  \multicolumn{1}{|r}{} &
  \multicolumn{1}{r}{} &
  \multicolumn{1}{r}{} &
  \multicolumn{1}{r}{} &
  \multicolumn{1}{r}{} &
  \multicolumn{1}{r}{} &
  \multicolumn{1}{r}{12} &
  \multicolumn{1}{r}{33} &
  \multicolumn{1}{r}{20} &
  \multicolumn{1}{r}{30} &
  \multicolumn{1}{r}{20} &
  \multicolumn{1}{r}{24} \\
\multicolumn{1}{l}{\hspace{1em}INDEM} &
  \multicolumn{1}{|r}{16} &
  \multicolumn{1}{r}{3} &
  \multicolumn{1}{r}{13} &
  \multicolumn{1}{r}{3} &
  \multicolumn{1}{r}{2} &
  \multicolumn{1}{r}{3} &
  \multicolumn{1}{r}{11} &
  \multicolumn{1}{r}{5} &
  \multicolumn{1}{r}{12} &
  \multicolumn{1}{r}{4} &
  \multicolumn{1}{r}{2} &
  \multicolumn{1}{r}{5} \\
\multicolumn{1}{l}{\hspace{1em}Otro} &
  \multicolumn{1}{|r}{2} &
  \multicolumn{1}{r}{} &
  \multicolumn{1}{r}{1} &
  \multicolumn{1}{r}{} &
  \multicolumn{1}{r}{} &
  \multicolumn{1}{r}{} &
  \multicolumn{1}{r}{} &
  \multicolumn{1}{r}{} &
  \multicolumn{1}{r}{} &
  \multicolumn{1}{r}{1} &
  \multicolumn{1}{r}{1} &
  \multicolumn{1}{r}{1} \\
\multicolumn{1}{l}{\hspace{1em}Total} &
  \multicolumn{1}{|r}{73.928} &
  \multicolumn{1}{r}{78.161} &
  \multicolumn{1}{r}{78.685} &
  \multicolumn{1}{r}{78.435} &
  \multicolumn{1}{r}{78.771} &
  \multicolumn{1}{r}{79.087} &
  \multicolumn{1}{r}{79.947} &
  \multicolumn{1}{r}{80.344} &
  \multicolumn{1}{r}{80.584} &
  \multicolumn{1}{r}{80.980} &
  \multicolumn{1}{r}{144.858} &
  \multicolumn{1}{r}{81.836} \\
\cline{1-13}
\end{tabular}

\end{tabular}
\end{center}
%\item \footnotesize Fuente : Registros administrativos del IPS.
\end{table}

Nota: CSML: Complemento por ajuste al 75\% del SML, JIP: Jubilación por
Invalidez Permanente, JIT: Jubilación por Invalidez Temporal, JO:
Jubilación Ordinaria, JP: Jubilación Proporcional, BAA: Beneficio
Adicional Anual, JCON:Jubilación por Convenios, J INT: Jubilación por
Intercajas, JNC-DP: Doc. Privados por Hacienda, LQVNP: Lo que en vida no
percibió, PDH: Pensión derechohabiente, Otro: Devolución por descuento
indebido, indemnización por nuevas nupcias, devolución de aportes,
indemnizaciones

\begin{table}[H]
\begin{center}
\caption{\bf{Evolución del monto promedio de los beneficios pagados en Jubilaciones y Pensiones en el año 2020 (Stock - En miles de Gs.).}}
\begin{tabular}{l|rrrrrrrrrrrrrrr}
\scriptsize
%\begin{tabular}{lllllllllllll}
\cline{1-13}
\multicolumn{1}{c}{} &
  \multicolumn{12}{|c}{Mes de pago} \\
\multicolumn{1}{c}{} &
  \multicolumn{1}{|r}{1} &
  \multicolumn{1}{r}{2} &
  \multicolumn{1}{r}{3} &
  \multicolumn{1}{r}{4} &
  \multicolumn{1}{r}{5} &
  \multicolumn{1}{r}{6} &
  \multicolumn{1}{r}{7} &
  \multicolumn{1}{r}{8} &
  \multicolumn{1}{r}{9} &
  \multicolumn{1}{r}{10} &
  \multicolumn{1}{r}{11} &
  \multicolumn{1}{r}{12} \\
\cline{1-13}
\multicolumn{1}{l}{Clasificación} &
  \multicolumn{1}{|r}{} &
  \multicolumn{1}{r}{} &
  \multicolumn{1}{r}{} &
  \multicolumn{1}{r}{} &
  \multicolumn{1}{r}{} &
  \multicolumn{1}{r}{} &
  \multicolumn{1}{r}{} &
  \multicolumn{1}{r}{} &
  \multicolumn{1}{r}{} &
  \multicolumn{1}{r}{} &
  \multicolumn{1}{r}{} &
  \multicolumn{1}{r}{} \\
\multicolumn{1}{l}{\hspace{1em}CSML} &
  \multicolumn{1}{|r}{331} &
  \multicolumn{1}{r}{495} &
  \multicolumn{1}{r}{493} &
  \multicolumn{1}{r}{493} &
  \multicolumn{1}{r}{493} &
  \multicolumn{1}{r}{493} &
  \multicolumn{1}{r}{492} &
  \multicolumn{1}{r}{484} &
  \multicolumn{1}{r}{481} &
  \multicolumn{1}{r}{479} &
  \multicolumn{1}{r}{478} &
  \multicolumn{1}{r}{478} \\
\multicolumn{1}{l}{\hspace{1em}JIP} &
  \multicolumn{1}{|r}{1.756} &
  \multicolumn{1}{r}{1.758} &
  \multicolumn{1}{r}{1.760} &
  \multicolumn{1}{r}{1.762} &
  \multicolumn{1}{r}{1.759} &
  \multicolumn{1}{r}{1.755} &
  \multicolumn{1}{r}{1.748} &
  \multicolumn{1}{r}{1.746} &
  \multicolumn{1}{r}{1.744} &
  \multicolumn{1}{r}{1.747} &
  \multicolumn{1}{r}{1.750} &
  \multicolumn{1}{r}{1.749} \\
\multicolumn{1}{l}{\hspace{1em}JIT} &
  \multicolumn{1}{|r}{2.158} &
  \multicolumn{1}{r}{2.201} &
  \multicolumn{1}{r}{2.221} &
  \multicolumn{1}{r}{2.204} &
  \multicolumn{1}{r}{2.330} &
  \multicolumn{1}{r}{2.299} &
  \multicolumn{1}{r}{2.296} &
  \multicolumn{1}{r}{2.313} &
  \multicolumn{1}{r}{2.226} &
  \multicolumn{1}{r}{2.176} &
  \multicolumn{1}{r}{2.363} &
  \multicolumn{1}{r}{2.317} \\
\multicolumn{1}{l}{\hspace{1em}JO} &
  \multicolumn{1}{|r}{4.453} &
  \multicolumn{1}{r}{4.474} &
  \multicolumn{1}{r}{4.487} &
  \multicolumn{1}{r}{4.506} &
  \multicolumn{1}{r}{4.517} &
  \multicolumn{1}{r}{4.537} &
  \multicolumn{1}{r}{4.551} &
  \multicolumn{1}{r}{4.561} &
  \multicolumn{1}{r}{4.575} &
  \multicolumn{1}{r}{4.580} &
  \multicolumn{1}{r}{4.597} &
  \multicolumn{1}{r}{4.608} \\
\multicolumn{1}{l}{\hspace{1em}JP} &
  \multicolumn{1}{|r}{1.800} &
  \multicolumn{1}{r}{1.811} &
  \multicolumn{1}{r}{1.820} &
  \multicolumn{1}{r}{1.823} &
  \multicolumn{1}{r}{1.830} &
  \multicolumn{1}{r}{1.837} &
  \multicolumn{1}{r}{1.847} &
  \multicolumn{1}{r}{1.851} &
  \multicolumn{1}{r}{1.855} &
  \multicolumn{1}{r}{1.867} &
  \multicolumn{1}{r}{1.870} &
  \multicolumn{1}{r}{1.876} \\
\multicolumn{1}{l}{\hspace{1em}BAA} &
  \multicolumn{1}{|r}{403} &
  \multicolumn{1}{r}{849} &
  \multicolumn{1}{r}{1.477} &
  \multicolumn{1}{r}{1.176} &
  \multicolumn{1}{r}{} &
  \multicolumn{1}{r}{} &
  \multicolumn{1}{r}{958} &
  \multicolumn{1}{r}{1.186} &
  \multicolumn{1}{r}{1.313} &
  \multicolumn{1}{r}{2.171} &
  \multicolumn{1}{r}{3.544} &
  \multicolumn{1}{r}{893} \\
\multicolumn{1}{l}{\hspace{1em}PDH} &
  \multicolumn{1}{|r}{1.540} &
  \multicolumn{1}{r}{1.537} &
  \multicolumn{1}{r}{1.539} &
  \multicolumn{1}{r}{1.541} &
  \multicolumn{1}{r}{1.545} &
  \multicolumn{1}{r}{1.546} &
  \multicolumn{1}{r}{1.547} &
  \multicolumn{1}{r}{1.545} &
  \multicolumn{1}{r}{1.546} &
  \multicolumn{1}{r}{1.548} &
  \multicolumn{1}{r}{1.550} &
  \multicolumn{1}{r}{1.553} \\
\multicolumn{1}{l}{\hspace{1em}JCON} &
  \multicolumn{1}{|r}{2.511} &
  \multicolumn{1}{r}{2.548} &
  \multicolumn{1}{r}{2.535} &
  \multicolumn{1}{r}{2.536} &
  \multicolumn{1}{r}{2.545} &
  \multicolumn{1}{r}{2.552} &
  \multicolumn{1}{r}{2.553} &
  \multicolumn{1}{r}{2.560} &
  \multicolumn{1}{r}{2.592} &
  \multicolumn{1}{r}{2.609} &
  \multicolumn{1}{r}{2.578} &
  \multicolumn{1}{r}{2.571} \\
\multicolumn{1}{l}{\hspace{1em}JINT} &
  \multicolumn{1}{|r}{} &
  \multicolumn{1}{r}{} &
  \multicolumn{1}{r}{1.187} &
  \multicolumn{1}{r}{1.187} &
  \multicolumn{1}{r}{1.187} &
  \multicolumn{1}{r}{1.107} &
  \multicolumn{1}{r}{1.107} &
  \multicolumn{1}{r}{1.107} &
  \multicolumn{1}{r}{1.107} &
  \multicolumn{1}{r}{1.107} &
  \multicolumn{1}{r}{1.150} &
  \multicolumn{1}{r}{1.222} \\
\multicolumn{1}{l}{\hspace{1em}LQVNP} &
  \multicolumn{1}{|r}{7.061} &
  \multicolumn{1}{r}{10.876} &
  \multicolumn{1}{r}{14.382} &
  \multicolumn{1}{r}{10.643} &
  \multicolumn{1}{r}{8.567} &
  \multicolumn{1}{r}{2.855} &
  \multicolumn{1}{r}{1.526} &
  \multicolumn{1}{r}{4.257} &
  \multicolumn{1}{r}{6.518} &
  \multicolumn{1}{r}{5.384} &
  \multicolumn{1}{r}{6.606} &
  \multicolumn{1}{r}{4.320} \\
\multicolumn{1}{l}{\hspace{1em}GF} &
  \multicolumn{1}{|r}{} &
  \multicolumn{1}{r}{} &
  \multicolumn{1}{r}{} &
  \multicolumn{1}{r}{} &
  \multicolumn{1}{r}{} &
  \multicolumn{1}{r}{} &
  \multicolumn{1}{r}{4.043} &
  \multicolumn{1}{r}{4.170} &
  \multicolumn{1}{r}{3.468} &
  \multicolumn{1}{r}{3.837} &
  \multicolumn{1}{r}{4.400} &
  \multicolumn{1}{r}{4.580} \\
\multicolumn{1}{l}{\hspace{1em}INDEM} &
  \multicolumn{1}{|r}{17.973} &
  \multicolumn{1}{r}{11.190} &
  \multicolumn{1}{r}{11.305} &
  \multicolumn{1}{r}{32.621} &
  \multicolumn{1}{r}{13.942} &
  \multicolumn{1}{r}{14.570} &
  \multicolumn{1}{r}{10.085} &
  \multicolumn{1}{r}{12.045} &
  \multicolumn{1}{r}{13.629} &
  \multicolumn{1}{r}{13.610} &
  \multicolumn{1}{r}{50.987} &
  \multicolumn{1}{r}{10.620} \\
\multicolumn{1}{l}{\hspace{1em}Otro} &
  \multicolumn{1}{|r}{251} &
  \multicolumn{1}{r}{} &
  \multicolumn{1}{r}{959} &
  \multicolumn{1}{r}{} &
  \multicolumn{1}{r}{} &
  \multicolumn{1}{r}{} &
  \multicolumn{1}{r}{} &
  \multicolumn{1}{r}{} &
  \multicolumn{1}{r}{} &
  \multicolumn{1}{r}{11.667} &
  \multicolumn{1}{r}{11.667} &
  \multicolumn{1}{r}{11.667} \\
\multicolumn{1}{l}{\hspace{1em}Total} &
  \multicolumn{1}{|r}{2.945} &
  \multicolumn{1}{r}{2.841} &
  \multicolumn{1}{r}{2.851} &
  \multicolumn{1}{r}{2.862} &
  \multicolumn{1}{r}{2.872} &
  \multicolumn{1}{r}{2.881} &
  \multicolumn{1}{r}{2.880} &
  \multicolumn{1}{r}{2.882} &
  \multicolumn{1}{r}{2.888} &
  \multicolumn{1}{r}{2.892} &
  \multicolumn{1}{r}{3.186} &
  \multicolumn{1}{r}{2.892} \\
\cline{1-13}
\end{tabular}

\end{tabular}
\end{center}
%\item \footnotesize Fuente : Registros administrativos del IPS. 
\end{table}
