\section{Marco Legal}

En este capítulo se presenta la descripción de las características
relevantes del Sistema Previsional Nacional, los antecedentes históricos
y el marco legal del Fondo de Jubilaciones y Pensiones administrado por
el Instituto de Previsión Social, en sus aspectos más resaltantes:
ámbito de aplicación, recursos, fuentes de financiamiento, prestaciones
que otorga y el análisis de las reglamentaciones vigentes en relación a
los aumentos que han sufrido los haberes jubilatorios respecto al
Salario Mínimo Le gal vigente.

\subsection{La previsión social en Paraguay}

En el Paraguay el origen histórico del Sistema de Jubilaciones y
Pensiones se remonta al año 1909 a través de la Ley de Organización
Administrativa de la Administración Central, donde se contemplaba una
cobertura a los funcionarios públicos permanentes de la Administración
Central, agentes de policía y militares, cuyas remuneraciones figuraban
en el Presupuesto General de Gastos de la Nación. Posteriormente este
régimen se convirtió en la Caja de Jubilaciones y Pensiones del
Ministerio de Hacienda, qu e agrupa a los empleados públicos (Magisterio
Nacional; Personal Militar de las Fuerzas Armadas de la Nación; Personal
Policial; Miembros del Poder Legislativo de la Nación; Magistrados y
funcionarios del Poder Judicial; y funcionarios de la Universidad
Nacional de Asunción) y algunas pensiones graciables (Veteranos de la
Guerra del Chaco).

En el año 1924 se crea la Caja de Jubilaciones y Pensiones de Empleados
Ferroviarios, sin embargo, no existía protección para el grueso de la
población trabajadora. Los trabajadores asalariados o contratados del
sector privado fueron incluidos dentro de l Régimen de Jubilaciones y
Pensiones en el año 1943, con la creación del Instituto de Previsión
Social (IPS). Esta institución se encarga no sólo de las Jubilaciones y
Pensiones, sino también ofrece un Seguro de Salud, así como las
prestaciones de un S eguro de Riesgos Profesionales (accidente de
trabajo y enfermedades profesionales).

Con el transcurso del tiempo fueron creándose nuevas Cajas de
Jubilaciones y Pensiones para diferentes gremios e instituciones, tales
como: la Caja de Jubilaciones y Pensiones de Empleados Bancarios (1951);
la Caja de la Administración Nacional de Elect ricidad (1968); la Caja
de Jubilaciones y Pensiones del personal de la Municipalidad de Asunción
(1978) y El Fondo de Jubilaciones y Pensiones para miembros del Poder
Legislativo de la Nación (1980).

Actualmente la Seguridad Social se encuentra estructurada en ocho Cajas
reguladoras que proveen prestaciones a sus respectivos colectivos:

\begin {enumerate}

\item Instituto de Previsión Social – IPS. 

\item Caja Paraguaya de Jubilaciones y Pensiones del Personal de la Itaipú Binacional.

\item El Fondo de Jubilaciones y Pensiones para miembros del Poder Legislativo de la Nación. 

\item La Caja de Seguros Sociales de Empleados y Obreros Ferroviarios. 

\item La Caja de Jubilaciones y Pensiones de Empleados de Bancos y Afines.

\item La Caja de Jubilaciones y Pensiones del Personal Municipal.

\item La Caja de Jubilaciones y Pensiones del Personal de la Administración Nacional de Electricidad (ANDE). 

\item El Fondo de Jubilaciones y Pensiones del Ministerio de Hacienda. 

\end{enumerate}

Todas las Cajas de Jubilaciones y Pensiones son financiadas a través de
contribuciones obligatorias sobre la nómina salarial que se fija como
porcentaje del mismo. Una parte la paga el asegurado, otra el empleador
y en algunos casos existe un aporte est atal.

También poseen diferencias muy grandes unas con otras en lo que respecta
a las prestaciones otorgadas y sistemas de administración. En lo
referente a su funcionamiento, algunas poseen elevadas pérdidas
operativas, las cuales en algunos casos son cubiert as con recursos
fiscales, constituyendo una carga para la sociedad en su conjunto.

Sólo el IPS brinda una cobertura integral, incluyendo prestaciones de
Salud y Riesgos Laborales. El resto de las Cajas se limita a otorgar
exclusivamente Jubilaciones y Pensiones.

El Sector Privado tiene escasa participación en el régimen de
Jubilaciones y Pensiones, ya que su incursión en esta área es
relativamente reciente, la adhesión es optativa, carece de una
legislación y supervisión adecuada. El sistema de capitalización i
ndividual y la figura jurídica de Mutuales son las características de
los principales Administradores de Fondos Privados en el mercado
paraguayo y representan una alternativa ante la ausencia de cobertura
para los trabajadores autónomos o independientes , personal gerencial y
empresarios.

Por Ley N° 4933/13 se autoriza la incorporación voluntaria de los
trabajadores independientes, empleadores, ama de casa y trabajadores
domésticos al seguro social por medio de la caja de jubilaciones,
empero, no se autoriza la incorporación de estos suj etos al sistema de
salud.

La Ley Nº 5741/16 establece la incorporación al Seguro Social del
Instituto de Previsión Social a los propietarios y/o responsables de las
Microempresas definidas en el artículo 4° de la Ley N°4457/12 ``para las
Micro, Pequeñas y Medianas Empresas (MIPYM ES)'', en forma obligatoria.
Esta Ley aún no fue reglamentada por lo que a la fecha este colectivo no
se halla inscripto en el Seguro Social administrado por el Instituto de
Previsión Social.

La Ley establece como autoridad de aplicación de la citada Ley al
Ministerio de Industria y Comercio, el que trabajará coordinadamente con
el Instituto de Previsión Social y el Ministerio de Trabajo, Empleo y
Seguridad Social en la reglamentación, mecan ismos de promoción e
incorporación gradual de las MIPYMES al Seguro Social del Instituto de
Previsión Social.

\subsection{Contexto legal}

En el año 1943 se crea el Instituto de Previsión Social, un organismo
autónomo con personería jurídica y patrimonio propio, que se rige por
las leyes vigentes y los reglamentos que dicta la propia institución. La
Dirección y Administración del Instituto están a cargo de un Consejo de
Administración, el cual está integrado por: un presidente y cinco
miembros que representan al Ministerio de Trabajo, Empleo y Seguridad
Social, al Ministerio de Salud Pública y Bienestar Social, a los
Empleadores, a los T rabajadores Asegurados y a los Jubilados y
Pensionados del Instituto, designados y supervisados por el Estado. Cada
uno de los representantes designados tendrá su respectivo suplente.

Posteriormente, en 1950 mediante el Decreto 1860/50 se establece la
naturaleza, objetivos, funciones, perfil jurídico y financiero del
mismo, configurándose como un organismo autónomo. En el año 1956 por la
Ley 375/56 se aprueba el citado Decreto y se e stablece las competencias
del Seguro Social, el cual debe cubrir los riesgos de Enfermedad no
Profesional, Maternidad, Accidentes del trabajo y Enfermedades
profesionales, Invalidez, Vejez y Muerte de los trabajadores asalariados
de la República del Par aguay.

En la Constitución de 1992 se consagra la obligatoriedad de la Seguridad
Social para los trabajadores dependientes y sus familias, y se promueve
la extensión de la misma a todos los sectores de la población. Asimismo,
se establece que los Sistemas de Se guridad Social pueden ser de
carácter público, privado o mixto, encontrándose en todos los casos bajo
supervisión estatal.

Si bien mediante la Ley 5115/13 se crea el Ministerio de Trabajo, Empleo
y Seguridad Social, al que corresponde la tutela de los derechos de los
trabajadores y las trabajadoras en materia de trabajo, empleo y
seguridad social, y una de sus competencias es la de ``propiciar la
elaboración, de programas y regímenes integrados de Seguridad Social del
sector público y del sector privado'', hasta el momento no existe un
marco regulatorio del Sistema de Seguridad Social en general ni del
Sistema de Jubilacion es y Pensiones en particular.

\subsection{Ámbito de aplicación del IPS}

De acuerdo con el Artículo 2º del Decreto Ley N° 1860/50, aprobado por
Ley N° 375/56, la cobertura de Jubilaciones y Pensiones del IPS está
dirigida a los trabajadores asalariados que presten servicios o ejecuten
una obra en virtud de un contrato de tra bajo, verbal o escrito,
cualquiera sea su edad y el monto de la remuneración que perciban.
También quedan incluidos en el régimen los aprendices y el personal de
los entes descentralizados del Estado o empresas mixtas.

Posteriormente se incorporaron al IPS otros colectivos, pero únicamente
al Seguro de Salud, como ser los docentes públicos (1958), los docentes
privados y el servicio doméstico (1965), los profesores universitarios
públicos y privados (1992), los docent es jubilados del Magisterio
Público (1999) y funcionarios activos y jubilados del Ministerio Público
- Fiscalía General del Estado - (2008), los Docentes Universitarios
Jubilados (2009), los Artistas y Cultores del Arte Independientes
(2010), los Propie tarios y/o Responsables de las Microempresas
definidas en el artículo 4° de la Ley N°4457/12 (2016). Faltando aún la
implementación para los dos últimos colectivos.

Hasta el año 2010 no se podía computar los aportes realizados a
diferentes Cajas Previsionales, lo que dificultaba completar el
requisito de años de aportes ante la ausencia de Jubilaciones
Proporcionales. Mediante la Ley 3856/09 de ``Intercajas'' se rec onocen
los aportes realizados a distintas Cajas Previsionales en el país, y
corresponde a cada una aportar proporcionalmente para la Jubilación del
Beneficiario.

A partir de la implementación de la Ley 4290/11, con 15 años de aportes
al Sistema Jubilatorio administrado por el IPS y 65 años de edad se
puede acceder a una Jubilación Proporcional, denominada así atendiendo
que el Haber Jubilatorio a percibir se det ermina como una proporción de
la Jubilación Ordinaria (con 25 años de aportes).

La Ley 4370/11 establece el Seguro Social (cobertura de Salud y
Jubilaciones) para los docentes dependientes de Instituciones Educativas
Privadas, con la salvedad que la base imponible es la suma total de las
remuneraciones realmente percibidas y que el Estado subsidiará, en
reconocimiento de los años de servicios un monto equivalente al aporte
jubilatorio no abonado en el período previo a la promulgación de la ley.
La Ley N° 5555/15 y el Decreto N° 5215/16, autorizaron los
procedimientos operativos y de implementación para determinar el ámbito
de aplicación, el cálculo y el pago del beneficio jubilatorio.

La Ley N° 4457/12, de las Micro, Pequeñas y Medianas Empresas, en su
artículo 45 autoriza al empleador el pago de los salarios sobre una base
no inferior al 80\% del Salario Mínimo Legal vigente durante los
primeros tres años desde su formalización, y e stableciendo la
obligatoriedad del Seguro Social del IPS en su artículo 47, se crea una
nueva base imponible del 80\% del Salario Mínimo Legal para las
referidas MIPYMES. Hasta la fecha, esta Ley no ha sido implementada.

La Ley 4933/13, permite la incorporación voluntaria de trabajadores
independientes, empleadores, amas de casa y trabajadores domésticos, al
fondo de Jubilaciones y Pensiones del Instituto de Previsión Social. Los
beneficiados del sistema voluntario son: trabajadores independientes
(quienes desempeñan habitualmente actividades lucrativas por cuenta
propia y no cuentan con personal asalariado a su cargo); los empleadores
(que en su función de empresa, negocio o actividad lícita de cualquier
clase, utili zan mediante un contrato de trabajo escrito o verbal, los
servicios de una o más personas); y amas de casa que realicen tareas
domésticas en su propio domicilio sin percibir remuneración alguna.

También eran beneficiarios del sistema voluntario los propietarios de
Micro, Pequeñas y Medianas Empresas, pero la Ley Nº 5741/16 regula la
incorporación al Seguro Social del Instituto de Previsión Social a los
propietarios y/o responsables de las Micro empresas definidas en el
artículo 4° de la Ley N°4457/12 ``para las Micro, Pequeñas y Medianas
Empresas (MIPYMES)'', en forma obligatoria. Hasta la fecha, esta Ley no
ha sido implementada.

Asimismo, la Ley Nº 5407/15 establece que las personas contratadas para
el trabajo doméstico cualquiera sea la modalidad del mismo, sean
incorporadas al Régimen General (cobertura de Salud y Jubilaciones) del
Seguro Social obligatorio del Instituto de P revisión Social. Esta
disposición se encuentra implementada y vigente desde el mes de
noviembre de 2015.

La Resolución MTESS N° 2660/2019 de fecha 30 de julio de 2019, regula la
inscripción de los trabajadores domésticos de tiempo parcial al Seguro
Social del Instituto de Previsión Social (IPS)\cite{METSS. (2019)}.

Mediante esta, los trabajadores domésticos, que trabajen entre 16 (diez
y seis) y 32 (treinta y dos) horas semanales podrán acogerse a los
beneficios del seguro social.

Los aportes asignados al seguro social correspondientes al servicio
doméstico en la modalidad de tiempo parcial serán distribuidos de la
siguiente manera:

\begin {itemize}

\item La cuota mensual del 9\% a cargo del trabajador calculado sobre la remuneración realmente percibida.

\item La cuota mensual del 14\% a cargo del empleador calculado sobre la remuneración realmente abonada.

\item El aporte patronal del 2.5\% calculado sobre la remuneración realmente abonada.

\end{itemize}

Además, los empleadores que tengan asegurados a sus trabajadores
domésticos en el régimen general doméstico podrán trasladarlos al
régimen parcial doméstico.

Para la inscripción al seguro, bajo la modalidad de tiempo parcial, el
empleador deberá presentar ante el IPS el contrato escrito suscrito
entre el mismo y el trabajador. Este es un requisito indispensable en
todos los casos.

De esta forma, los asegurados se pueden clasificar en dos tipos de
regímenes: General, Parcial y Especial. Los trabajadores incluidos en el
Régimen General y Parcial cotizan tanto para los beneficios de
jubilaciones y pensiones como los de salud, mientr as que los de los
Regímenes Especiales sólo acceden al seguro de salud.

\textbf{Régimen General:}

\begin {itemize}

\item Asalariados dependientes del sector privado; 

\item Funcionarios y obreros de entes descentralizados del Estado y de las empresas de economía mixta;

\item Funcionarios, empleados y obreros de la Administración Nacional de Electricidad (ANDE)\footnote{Si bien se los considera del Régimen General por aportar al Seguro de Salud y al Fondo de Jubilaciones y Pensiones, realizan aportes diferenciales, por
que el IPS sólo otorga un porcentaje de las Jubilaciones y la diferencia es complementada por la Caja de la ANDE.}

\item Docentes privados de los niveles básico, medio y superior, formal y no formal;

\item Trabajadores del servicio doméstico

\end{itemize}

\textbf{Regímen Parcial:} Trabajadores domésticos de tiempo parcial

\textbf{Regímenes Especiales:}

\begin{itemize}


\item Magisterio público: comprende a los profesores de la enseñanza básica, media, universitaria, técnica y de idiomas;

 \item Artistas en general sin relación de dependencia\footnote{ Hasta la fecha no ha sido implementado.};
 
 \item Jubilados y pensionados del IPS;
 
 \item Jubilados del magisterio público;
 
 \item Veteranos de la Guerra del Chaco (o su esposa o concubina).
 
\end{itemize}

\subsection{Recursos y Financiamiento}

El principal recurso con que cuenta el IPS son las cotizaciones a cargo
de trabajadores, empleadores, jubilados y pensionados que deben pagar
mensualmente. Los aportes están determinados por ley y sólo pueden
modificarse mediante otra ley.

La ley también establece un aporte del Estado equivalente al 1,5\% de
los salarios sobre los cuales cotizan los empleadores. Sin embargo, este
aporte no se ha hecho efectivo, y el IPS lo registra contablemente como
un activo\footnote{Balance de Comproba
ción de Saldos y Variaciones al 31/12/2017.}

El resto de los recursos se integra con la renta de las inversiones del
Fondo de Jubilaciones y Pensiones, el ingreso por recargos y multas
aplicadas de conformidad con las disposiciones legales, el ingreso por
las prestaciones médicas y servicios urgen tes en hospitales del
Instituto a personas no aseguradas conforme a tarifas establecidas por
el Consejo de Administración del Instituto, los legados y donaciones que
hicieren al Instituto y cualquier otro ingreso que obtenga el Instituto
no especificado en la ley.

La legislación\footnote{Art. 20 de la Ley 1860/50.} establece que
\textit{"Ninguna cotización será inferior a la que corresponda al salario o sueldo mínimos legalmente fijados, aunque se trate de Aprendices que no reciben salario en dinero"},
lo cual se reglamentó a través de la Resolución del Consejo de
Administración Nº 069-017/05 que establece:

\textit{"A los efectos de la presente Resolución, "salario mínimo" se refiere a "salario mínimo legal" mensual para actividades diversas no especificadas de la Capital establecido por la correspondiente Resolución reglamentaria del Ministerio de Justici
a y Trabajo (actualmente Ministerio de Trabajo, Empleo y Seguridad Social) vigente, salvo los siguientes casos que se regirán por la actividad específica prevista en la misma Resolución:}

\begin {itemize}

\item\textit{El salario previsto para trabajo de menores y aprendices, los cuales no serán inferior al 60\% del salario mínimo para actividades diversas no especificadas, regulados en la Ley 213/93 "Código del Trabajo".}

\item\textit{Establecimientos ganaderos, categoría "A" y "B", a los cuales deberá adicionarse el 20\% (veinte por ciento), en concepto de avaluación pecuniaria de las regalías recibidas por el trabajador. Se entenderá que los establecimientos ganaderos 
son de categoría B, salvo prueba en contrario por parte del empleador}.

\item\textit{Chofer cobrador, conforme a la Resolución reglamentaria del Ministerio de Justicia y Trabajo (actualmente Ministerio de Trabajo, Empleo y Seguridad Social), vigente.}

\end{itemize}

\textit{En caso de trabajo a destajo o a jornal, el salario mínimo imponible será el equivalente a 25 (veinte y cinco) jornales mínimos para actividades diversas no especificadas para la Capital, fijado por Resolución del Ministerio de Justicia y Trabaj
o (actualmente Ministerio de Trabajo, Empleo y Seguridad Social).}

\textit{Las Entidades Públicas podrán justificar mediante Nota Institucional los casos especiales que no se adecuen a la presente reglamentación, para cuyo efecto se emitirán planillas complementarias que serán sometidas a consideración del Consejo de A
dministración en cada caso".}

La Resolución C.A. Nº 076-037/05 del 4 de octubre de 2005, modifica y
complementa la resolución C.A. Nº 069-017/05 y establece como base
mínima sobre la cual deben los empleadores aportar por los jornaleros y
trabajadores a destajo, el importe correspon diente al total de 18 (diez
y ocho) jornales mínimos para actividades diversas no especificadas de
la Capital fijado por Resolución del Ministerio de Justicia y Trabajo
(actualmente Ministerio de Trabajo, Empleo y Seguridad Social).

Por otro lado, también se establece que los descuentos o retenciones de
cuotas o cotizaciones que realizan los empleadores a sus empleados no
pueden exceder del 9\% de los sueldos efectivamente pagados. Por lo
tanto, los empleadores tienen a su cargo la integración de las
diferencias para completar las que correspondan a los mínimos
mencionados.

En caso de que los trabajadores perciban un salario mensual inferior al
salario mínimo, establecido para actividades diversas no especificadas
de la Capital, aún en el caso de que sea proporcional al mínimo legal,
la diferencia que surja entre el 9\% (n ueve por ciento) descontado al
trabajador del salario real percibido, y lo restante para alcanzar el
25,5\% (veinte y cinco y medio por ciento) del salario mínimo legal,
para la actividad correspondiente, será integrada por el empleador. Este
mismo crit erio será aplicado aún en los casos en que el porcentaje de
aporte sea distinto.

Es importante mencionar que en Paraguay el aguinaldo se abona en el mes
de diciembre de cada año. El mismo equivale a la doceava parte de las
remuneraciones devengadas durante el año calendario a favor de los
trabajadores y no se realizan aportes al Sis tema de Seguridad Social
sobre el mismo.

El Artículo 7° de la Ley N° 98/92 incorpora la forma de distribuir el
superávit del ejercicio que, en caso de existir, debe realizarse de la
siguiente manera:

\begin {itemize}

\item\textbf{Reservas Técnicas}: Fondo Común de Jubilaciones y Pensiones: 70\% (setenta por ciento).

\item\textbf{Fondo de Previsiones}: Para ajuste de Jubilaciones y Pensiones: 25\% (veinticinco por ciento) y    Para Imprevistos: 5\% (cinco por ciento). 

\end{itemize}

El mismo artículo también establece que el Consejo de Administración
puede disponer del uso de los Fondos de Previsiones para imprevistos,
cuando las necesidades o circunstancias especiales así lo justifiquen.

\subsubsection{Distribución de los Recursos}

El Art. 95 de la Constitución Nacional establece que
\textit{Los recursos financieros de los seguros sociales no serán desviados de sus fines específicos y; estarán disponibles para este objetivo, sin perjuicio de las inversiones lucrativas que puedan a
crecentar su patrimonio} \cite{constitucion}.

Los recursos se distribuyen en tres fondos: Fondo de Jubilaciones y
Pensiones; Fondo de Salud; y Fondo de Administración. La obtención de
los recursos financieros de cada uno de ellos es independiente y
autónoma de los demás, recibiendo solamente los i ngresos especificados
en la ley, no pudiendo transferirse recursos de un fondo a otro, así
como ninguna clase de operaciones que tengan por consecuencia el empleo
de los recursos en forma distinta a la determinada por la Ley.

Los recursos para el Régimen General se encuentran establecidos por ley
(Carta Orgánica del IPS), estableciéndose el pago del 23\% sobre la base
imponible, donde el 9\% es descontado al empleado (aporte) y el 14\% es
pagado por el empleador (contribució n). Lo recaudado se asigna a los
distintos fondos, donde el 12,5\% sobre la base imponible corresponde
para el Fondo de Jubilaciones y Pensiones, el 9\% para las prestaciones
de Salud y el 1,5\% para los gastos de Administración. Lo que representa
del total recaudado en concepto de las contribuciones sociales, el 54,3
\% se asigna al Fondo de Jubilaciones, el 39,2\% al Fondo de Enfermedad
y Maternidad y el 6,5\% al Fondo de Administración.

\subsection{Prestaciones}

\subsubsection{Jubilaciones Ordinarias}

Originalmente la Ley N°375/56 establecía como requisitos para acceder a
la Jubilación Ordinaria 15 años de aportes y 60 años de edad con un
beneficio del 42,5\% del salario promedio de los últimos 3 años,
incrementándose 1,5\% por cada año en exceso de 15. Este cálculo
implicaba que, para una carrera de 30 años, el haber jubilatorio sería
del 65\% del salario indicado.

En 1973, la Ley N° 430/73 generó un cambio sustancial, creando un
régimen complementario, que con un 5\% de aportes adicionales, se
otorgaban prestaciones que sumadas a la Jubilación Ordinaria podían
alcanzar el 100\% del salario promedio de los últimos 36 meses.

La reforma del año 1992, mediante la sanción de la Ley N°98/92, absorbió
al Fondo de Jubilaciones y Pensiones Complementario creado por la Ley
N°430/73, unificando así ambos Fondos.

Actualmente, según la Ley N° 98/92, para acceder a la Jubilación
Ordinaria, se establece como requisitos que el asegurado haya cumplido
60 años de edad y tenga como mínimo 25 años de servicios reconocidos,
correspondiéndole en este caso un haber del 100 \% del promedio de los
salarios de los 36 últimos meses anteriores al último aporte. También se
puede acceder a esta prestación con 55 años cumplidos y 30 años como
mínimo de servicios reconocidos, en cuyo caso le corresponde al
asegurado el 80\% del p romedio de salarios citado anteriormente. Este
porcentaje aumentará a razón del 4\% por cada año que sobrepase los 55
años de edad, en el momento de solicitarlo, hasta los 59 años de edad.

Asimismo, el Art. 6° de la Ley N° 98/92, establece que para el caso del
personal de la ANDE (Administración Nacional de Electricidad), los
requisitos son 750 semanas de aportes y 60 años, abonándose en este caso
el 42,5\% del promedio de salario de los últimos 36 meses, e
incrementándose a razón del 1,5\% por cada 50 semanas que sobrepase el
requisito mínimo, hasta el máximo establecido en la Ley. En este caso
tienen unas alícuotas de aportes obrero-patronal diferentes a la del
Régimen General.

Como la norma original establecía como requisito un mínimo 750 semanas
de cuotas, y la misma se sustituyó al promulgarse la Ley N° 98/92, se
podía dar el caso de asegurados que habiendo reunido dicho requisito se
presentaren posteriormente a la derogaci ón de dicha disposición a
solicitar la Pensión de Vejez. Por este motivo, el IPS mediante
Resolución CA N° 2574/97, estableció el 28 de febrero de 1999 como la
fecha límite hasta la cual podrían expedir la solicitud con dichos
requisitos, independientem ente de la fecha en que se presenten al
Instituto a peticionar dicha Pensión.

Los salarios que conforman la base del cálculo inicial no se actualizan,
sino que se toman en términos nominales, excluyendo el último aporte
realizado al IPS. Una vez iniciado el cobro, los haberes obtienen
ajustes anuales que autoriza el Consejo de Ad ministración, de acuerdo
con el índice del costo de vida medido oficialmente por el Banco Central
del Paraguay a través del Índice de Precios al Consumidor, el que debe
abonarse con efecto retroactivo al mes de
enero\footnote{Segunda parte del Art. 26 d
e la Ley 1860/50, modificado por el Art. 2 de la Ley 98/92}.

\subsubsection{Reconocimiento de Servicios Anteriores (RSA)}

Para que los aportes anteriores a febrero de 1974 sean considerados para
los beneficios de la Ley N° 430/73 (hasta un 100\% del salario
promedio), los mismos deben ser verificados por el IPS correspondiendo
pagar el diferencial de tasa de aporte del 5\% sobre los mismos. Con
esta finalidad, primero deben ser actualizados, determinando luego el
valor de los aportes no pagados (deuda por RSA), que deberá pagarse
mediante los descuentos que se le practiquen del beneficio de la
respectiva jubilación. La L ey N° 2755/05 reabrió un período
complementario para este reconocimiento de servicios. La vigencia del
periodo de admisión de solicitudes concluyó el 18 de octubre de 2007.

La Ley Nº 4290/11 habilitó de manera permanente el derecho a solicitar
Reconocimiento de Servicios Anteriores (RSA) al mes de febrero de 1974,
y estableció que cuando el RSA permita a una persona que ya cuenta con
la edad requerida, completar el requisi to de antigüedad exigido por la
legislación respectiva a fin de lograr una jubilación, esta se concederá
conforme a las siguientes disposiciones:

\begin{enumerate}[label=\alph*.]

\item Asegurado pasivo con solicitud de jubilación anterior a la vigencia de la presente Ley: el beneficio se concederá a partir de la fecha de formulación de la nueva solicitud de reconocimiento de servicios anteriores.

\item Asegurado pasivo sin solicitud de jubilación anterior a la vigencia de la presente Ley: el beneficio se concederá a partir de la fecha de solicitud de la respectiva jubilación.

\item Asegurado activo con solicitud de jubilación anterior a la vigencia de la presente Ley: el beneficio se concederá a partir del primer día del mes siguiente al de su retiro del trabajo.

\end{enumerate}

Los asegurados que antes de la vigencia de esta Ley, estuvieren ya
jubilados conforme a las disposiciones de la Ley Nº 430/73 y de la
Resolución C.A. Nº 2574/97, no podrán solicitar reconocimiento de
servicios anteriores en virtud de esta Ley.

\subsubsection{Continuidad en el seguro}

La Ley N°3404/07 introdujo una modificación importante para aquellos
asegurados que se retiran de su trabajo y no tienen reunidos los
requisitos para obtener una Jubilación Ordinaria. La norma establece que
en estos casos, el afiliado puede solicitar su continuidad en el Seguro
al solo efecto de la Jubilación Extraordinaria, con lo cual debe
realizar mensualmente un aporte obligatorio del 12,5\% del promedio de
los 36 últimos salarios cotizados con anterioridad al cese laboral.
Adicionalmente se estab lece que, si habían transcurrido más de dos años
del desembolso de aquellas cotizaciones, los salarios tomados para el
cálculo debían ser actualizados conforme a las variaciones del IPC y del
SML, ponderados por partes iguales, 50\% cada variable.

La Ley N°4290/11 aclara el alcance de la modalidad de cotización
establecida en la Ley N°3404/07, y de esta manera permite reunir los
requisitos de edad y antigüedad establecidos para todas las modalidades
jubilatorias otorgadas por el Instituto de Prev isión Social, con
excepción de las causadas por Invalidez, por Enfermedad Común e
Invalidez por Accidente del Trabajo o Enfermedad Profesional.

\subsubsection{Jubilación Proporcional}

La Ley Nº 4290/11 permite a los ciudadanos, hacer valer los aportes
realizados antes del año 1974, a efectos de completar la antigüedad
requerida por el Seguro Social. Esta Ley adicionalmente introduce la
Jubilación Proporcional como uno de los benefici os jubilatorios a ser
otorgados por el IPS.

A la Jubilación Proporcional tendrá derecho el asegurado que se
encuentre retirado de la actividad laboral, haya cumplido 65 (sesenta y
cinco) años de edad y tenga 750 (setecientos cincuenta) semanas de
cuotas como mínimo.

\subsubsection{Jubilación por Invalidez}

El Art. 61° de la Ley N° 98/92, establece que para acceder a una
Jubilación de Invalidez, además de la declaración de la invalidez por
parte de una comisión médica del IPS, el asegurado debe tener por lo
menos 150 semanas de cotizaciones y menos de 55 a ños de edad, o entre
150 a 250 semanas de cotizaciones y menos de 60 años, o entre 250 a 400
semanas de cotizaciones y menos de 65 años.

El monto de esta prestación es equivalente al 50\% del salario mensual
promedio de los 36 últimos meses anteriores a la declaratoria de
invalidez, más el 1,5\% de dicho monto, por cada 50 semanas de
cotizaciones que sobrepasen las 150 semanas de aportes , hasta totalizar
el 100\%.

El Fondo de Jubilaciones y Pensiones también financia los beneficios de
la invalidez provenientes de accidentes laborales o enfermedades
profesionales. El haber de esta prestación se determina en función de
una tabla que observa dos variables, la antigü edad del trabajador y el
porcentaje de pérdida de la capacidad del trabajo.

\subsubsection{Pensión por muerte}

Según Ley N° 2263/03, las pensiones se otorgan en el caso de
fallecimiento de:

\begin{enumerate}[label=\alph*.]

\item Asegurados activos con derecho para acceder a una jubilación.

\item Asegurados activos con un mínimo de 750 semanas de aportes sin tener la edad mínima para su jubilación. 

\item Jubilados

\end{enumerate}

El haber de la pensión equivale al 60\% del importe de la jubilación que
percibía o que le hubiera correspondido al causante. Los
derechohabientes son en orden excluyente:

\begin{enumerate}[label=\alph*.]

\item La viuda/concubina o el viudo/concubino con los hijos solteros hasta la mayoría de edad, y los incapacitados (sin límite de edad) y declarados tales por una Junta Médica del Instituto, en cuyo caso la mitad de la pensión corresponde a la viuda/con
cubina o el viudo/concubino, y la otra mitad a los hijos por partes iguales.

\item La viuda/concubina o el viudo/concubino menor de 40 años, le corresponderá una indemnización equivalente a 3 anualidades de la pensión que le hubiera correspondido. 

\item Los hijos huérfanos hasta la mayoría de edad; los hijos incapacitados y declarados tales por un Junta Médica del Instituto, por partes iguales la totalidad de la pensión. 

\item Los padres, siempre que hayan vivido bajo la protección del causante, en partes iguales. De sobrevivir uno de ellos recibirá el total de la pensión.

\end{enumerate}

Las pensiones indicadas en los incisos a) y c), aumentarán
proporcionalmente a medida que los beneficiarios concurrentes dejen de
tener derecho a ellas. La pensión de los hijos incapacitados se pagará
mientras dure la incapacidad de los mismos.

Para que la concubina o el concubino tengan derecho a la pensión deben
haber vivido voluntariamente en relación pública de notoriedad, estable
y singular, como mínimo durante 2 años si tuvieren hijos comunes y 5
años si no los tuvieren, y además estar i nscripta o inscripto en los
registros del IPS.

El derecho de percibir la pensión se adquiere desde la fecha del
fallecimiento del asegurado o de la asegurada, y se extingue si la
persona contrae matrimonio o vuelve a vivir en concubinato, recibiendo
en tales casos por una única vez la suma equivalen te a 2 anualidades de
la pensión.

Cuando el asegurado fallecido tuviere menos de 750 semanas de aportes,
se otorga a sus herederos o beneficiarios, un subsidio en dinero por una
sola vez equivalente a un mes de salario por cada año de antigüedad que
tuviere el asegurado. A dicho efecto se tomará como base el Salario
Mínimo Legal vigente para actividades diversas no especificadas en la
Capital de la República.

Si no existieren heredero o beneficiario, se abonará a quien o quienes
justifiquen haber realizado los gastos fúnebres correspondientes, hasta
un monto equivalente a 75 (setenta y cinco) Jornales Mínimos
establecidos para actividades diversas no especif icadas en la Capital
de la República. Cuando posteriormente apareciera algún heredero o
beneficiario, el monto de los gastos se descontará de la pensión o del
subsidio, en su caso.

\subsubsection{Beneficio adicional}

Las leyes N° 532/94 y Nº 731/95, establecen que los beneficiarios de
jubilaciones y pensiones perciban como beneficio adicional, un décimo
tercer salario (aguinaldo), equivalente a la doceava parte de las
remuneraciones devengadas durante el año calenda rio, el cual se abona
en el mes de noviembre. El Consejo de Administración tiene la potestad
para otorgarlo o no en función a los estudios actuariales que se
realicen y a las posibilidades financieras. En la práctica se ha estado
concediendo en forma re gular.

\subsubsection{Haber mínimo y máximo}

La Ley 4426/11 establece que el haber mínimo jubilatorio y de pensiones
para los asegurados del Instituto de Previsión Social no podrá ser
inferior al 33\% (treinta y tres por ciento) del salario mínimo vigente
para actividades diversas no especificadas , el cual deberá actualizarse
anualmente de acuerdo con el Índice de Precios al Consumidor declarado
por el Banco Central del Paraguay.

Por Resolución CA N° 004 - 001/2020 del 17 de enero de 2020, se autoriza
el incremento del Haber Mínimo Jubilatorio de los Jubilados y
Pensionados del Instituto de Previsión Social y se establecen las reglas
de aplicación. La mencionada Resolución esta blece el aumento del Haber
Jubilatorio del 33\% (treinta y tres por ciento) al 50\% (cincuenta por
ciento) del salario mínimo vigente para actividades diversas no
especificadas, que será incrementado por los efectos de la suba del
salario mínimo vigent e\cite{IPS. (2020)}.

Los beneficiarios de este incremento serán los que actualmente perciben
el Haber Mínimo Jubilatorio del 33\% (treinta y tres por ciento) del
salario mínimo legal y aquellos que se encuentran comprendidos entre el
34\% (treinta y cuatro por ciento) y 49\\
\% (cuarenta y nueve por ciento) del salario mínimo vigente conforme a
la siguiente clasificación de beneficiarios:

\begin{enumerate}[label=\alph*.]

\item Beneficiarios por retiro de vejez.

\item Beneficiarios de Pensiones por muerte.

\item Beneficiarios de Jubilaciones por invalidez de carácter definitivo.

\end{enumerate}

Así mismo, se establece que quedan excluidos de este incrementos son:

\begin{enumerate}[label=\alph*.]

\item Los Beneficiarios del Sistema Intercajas, Ley 3856/09, 

\item Los Beneficiarios de Convenios Internacionales,

\item Los Beneficiarios de la Ley 4370/11 del 13 de julio de 2011 y sus modificatorias. 

\end{enumerate}

Dicha Resolución establece, así mismo, que la pensión por muerte
derivada del fallecimiento de los beneficiarios mencionados arriba,
deberán ajustarse al nuevo Haber Mínimo Jubilatorio.

Esta nueva disposición rige desde Febrero de 2020 y no será de
aplicación retroactiva en ningún caso.

En fecha 16 de febrero de 2021, se autoriza, mediante Resolución CA N°
017 - 001/2021, el incremento del Haber Mínimo Jubilatorio de los
Jubilados y Pensionados del Instituto de Previsión Social y se
establecen las reglas de aplicación. La mencionada Re solución establece
el aumento del Haber Jubilatorio del 50\% (cincuenta por ciento) al 75\%
(setenta y cinco por ciento) del salario mínimo vigente para actividades
diversas no especificadas, que será incrementado por los efectos de la
suba del salario mínimo vigente\cite{IPS. (2021)}.

Este haber se irá incrementando automáticamente en relación al aumento
del Salario Mínimo Vigente.

Los beneficiarios de este nuevo Haber Mínimo Jubilatorio serán los que
se encuentren percibiendo el Haber Mínimo Jubilatorio del 50\%
(cincuenta por ciento) y cuyos haberes sean infereiores al 75\% (setenta
y cinco por ciento) del salario mínimo vigente . El monto de la
prestación mínima para el año 2021 es de Gs. 1.716.993 (calculado en
base al salario mínimo de Gs.2.289.324).

Además, según la mencionada Resolución, quedan excluidos de este
incremento:

\begin{enumerate}[label=\alph*.]

\item Los Beneficiarios por muerte.

\item Los Beneficiarios de Jubilaciones por Invalidez no definidas.

\item Los Beneficiarios del Sistema Intercajas, Ley 3856/09.

\item Los Beneficiarios de Convenios Internacionales.

\item Los Beneficiarios de la Ley 4370/11 del 13 de julio de 2011 y sus modificatorias.

\end{enumerate}

El haber máximo quedó establecido por la Ley 1286/87, en cuyo Art. 6º
indica que el mismo, en el momento de la liquidación inicial, no
sobrepasará el equivalente a 250 (doscientas cincuenta) veces el valor
del salario mínimo diario vigente para activid ades no especificadas en
la capital de la República, el cual equivale aproximadamente a 8
salarios mínimos.

El artículo mencionado, fue modificado por el Art. 3 de la Ley 98/92,
quedando como sigue:
\textit{El máximo de cualquier jubilación mensual en virtud a esta ley en el momento de la liquidación inicial, no sobrepasará el equivalente de 300 (trescientas)
 veces el valor del jornal mínimo vigente para actividades diversas no especificadas en la Capital de la República, equivalente aproximadamente a 10 salarios mínimos}.
