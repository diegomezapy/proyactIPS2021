\subsection{Evolución y Composición del Portafolio de Inversiones}

En esta sección se presenta la composición y evolución de la cartera
financiera del Fondo de Jubilaciones y Pensiones a partir de la
información suministrada por la Unidad de Análisis Técnico, Financiero e
Inmobiliario de la Dirección del Inversiones de l IPS, Boletín de
Inversiones a diciembre
2020\footnote{https://portal.ips.gov.py/sistemas/ipsportal/archivos/archivos/1613754264.pdf}
e Informe de Gestión.

En la actualidad, el portafolio de inversiones cuenta con los siguientes
activos:

\begin{enumerate}
\item \textbf{Inversiones financieras:} Certificados de Depósito de Ahorro (CDA), Bonos, Títulos de Crédito, Préstamos con fianza.
\item \textbf{Préstamos a funcionarios, jubilados y pensionados del IPS} a través del Dpto. de Caja de Préstamo.
\item \textbf{Caja y equivalentes:} Caja de Ahorro a la vista y Cuenta Corriente.  
\item \textbf{Bienes inmuebles:} Inmuebles de renta.
\end{enumerate}

Al mes de diciembre del año 2020 el portafolio de inversiones expresado
en guaraníes asciende a \(\G\)
14.024.054.649.256\footnote{El portafolio total expresado en dólares americanos asciende a 2.032.439.287, según cotización diaria referencial del BCP, a
l 30/12/2020, igual a $\G$ 6.900,11.}.

La mayor parte de los recursos del IPS se encuentran distribuidos en
Inversiones Financieras con una participación del 66\%, seguido por Caja
y equivalentes, los cuales representan el 16\% del total del portafolio,
luego con el 12\% se encuentran los pr éstamos que la Institución otorga
a sus funcionarios, jubilados y pensionados a través de la Caja de
Préstamos, que totalizan \(\G\) 1,71 billones, en términos de saldo de
cartera; y por último, Inmuebles con una participación del 6\%.

El monto de los activos que componen el portafolio, expresado en
guaraníes y su respectiva participación dentro del total de la cartera,
se aprecia en la Tabla XX, en tanto que la evolución del portafolio se
observa en la Figura XX:

\textbf{***Tabla xxx - Composición del portafolio por tipo de activo, expresado en guaraníes . A diciembre 2020***}

\textbf{***Figura xxx - Evolución del monto total del portafolio, expresado en guaraníes. Periodo 2015-2020***}

La rentabilidad promedio de las inversiones en guaraníes se ubica en
torno al 6,94\%, en tanto que en dólares americanos se posiciona
alrededor del 5,72\%. El plazo promedio de las inversiones es de 4 años.

\textbf{***Tabla 14 - Rendimiento y plazo promedio del portafolio por activo y moneda. A diciembre 2020***}

Dentro de las inversiones financieras, la principal inversión la
componen los Certificados de Depósito de Ahorro (CDA´S) con el 81,8\%,
dicho porcentaje representa más de USD 1.092 millones, distribuidos en
15 entidades financieras de nuestro país. Segu idamente, se encuentran
los bonos, principalmente de la Agencia Financiera de Desarrollo (AFD),
con el 12,4\% y finalmente los Títulos de crédito y Préstamos con fianza
con el 5,5\% y 0,3\%, respectivamente.

\textbf{***Figura xxx – Distribución de las inversiones financieras. Diciembre 2020}

Teniendo en cuenta que los CDA's en moneda local representan más del
50\% dentro del portafolio de inversiones del Instituto en guaraníes, se
visualiza que el rendimiento promedio ponderado del portafolio del IPS
tiene un comportamiento similar al prome dio de las tasas pasivas de
CDA´s del sistema bancario (mayores a 365 días), siendo el rendimiento
del IPS superior al del sistema al cierre del último año.

\textbf{***Figura xxx – Evolución de tasas (portafolio en guaraníes): rendimiento promedio ponderado del IPS vs. Tasas Pasivas Sistema Bancario vs. CDA Sistema Bancario***}

Al observar la evolución del rendimiento promedio ponderado del
portafolio de IPS en dólares americanos con las tasas pasivas del
sistema bancario, se constata que el rendimiento del IPS se ha mantenido
por encima del sistema bancario, tanto en lo que r especta al promedio
ponderado total como respecto de las tasas pasivas promedio de CDA's en
moneda extranjera, puesto que el 92\% de los activos en dólares
americanos corresponden a CDA's.

\textbf{***Figura xxx  – Evolución de tasas (portafolio en dólares americanos): rendimiento promedio ponderado del IPS vs. Tasas Pasivas Sistema Bancario vs. CDA Sistema Bancario***}
