\subsubsection{Relación Activos/ Pasivos en IPS}

Una variable muy utilizada como primer indicador de referencia para los
Sistemas Previsionales es la relación entre Activos y Pasivos. Los
países que cuentan con Sistemas Previsionales de Beneficio Definido
financiados mediante Reparto\footnote{Sistema 
de Reparto es también conocido como PAY-AS-YOU-GO.} monitorean con
rigurosos estudios actuariales que la promesa de pagar determinados
montos de haber previsional pueda ser cumplida, para lo cual es
sumamente importante contar con una gran cantidad de trabajadores
activos que sustenten a los jubilados y pensionados.

En términos simples, suponiendo que todos los trabajadores perciben la
misma remuneración y cada uno aporta el 20\% de su remuneración y se
pretende que el haber jubilatorio sea equivalente al 80\% del salario,
la cantidad de trabajadores por cada benef icio jubilatorio debe ser
igual a 4 (ya que con 20\% de la remuneración de cuatro aportantes se
puede financiar una jubilación equivalente al 80\% del salario
promedio).

Bajo el supuesto de que todos los trabajadores cotizantes al IPS
perciben la misma remuneración, y esta es igual o mayor al SML, se
necesitarían como mínimo 8 trabajadores activos para financiar la
jubilación de una persona que se retira percibiendo el 100\% del
promedio de sus últimas 36 remuneraciones, habiendo sido éstas iguales
al SML.

En la Tabla xx se aprecia la evolución anual de la relación entre la
cantidad de activos sobre la cantidad de pasivos que, en el caso del IPS
se ha mantenido muy alta y prácticamente constante durante los últimos 5
años. En este periodo dicha relación h a permanecido en torno a 11
activos por cada pasivo. El Gráfico xx ilustra la evolución de este
indicador y muestra con mayor claridad la tendencia, principalmente
caracterizada por una leve disminución observada en los 5 años
analizados.

Precisamente en el mes de noviembre del año 2015 fueron incorporados al
Régimen General los trabajadores domésticos, lo que produjo un aumento
cercano al 4\% en la cantidad de cotizantes (aproximadamente 20.000
trabajadores). Sin embargo, esta incorpora ción de nuevos cotizantes no
se mantuvo en los años posteriores, pues como puede notarse, la relación
volvió a descender a los niveles anteriormente observados. Cabe señalar
que, este indicador muestra la relación bruta, ya que no tiene en cuenta
el por centaje de aporte, ni los salarios, ni los niveles de haber
jubilatorio.

\textbf{***agregarla tabla de Relación entre la cantidad de trabajadores activos y pasivos y entre los salarios y beneficios promedios***}

\textbf{***agregar el gráfico Evolución anual de la cantidad de activos sobre pasivos***}

A su vez, en la misma Tabla xx se puede observar la evolución de la
relación entre el salario promedio del trabajador activo y el monto
promedio de los beneficios pagados en J y P. En el año 2013 la relación
era de 1,31; es decir el salario promedio del activo era 31\% superior
respecto al beneficio promedio del pasivo. Para el año 2020 esta
relación disminuyó a xx lo cual puede interpretarse como un acercamiento
sustancial entre el salario promedio y el beneficio promedio. La
tendencia queda ilustrad a en el Gráfico xx y deja claro cómo los
beneficios promedios aumentan a un ritmo mayor que los salarios
promedios. Desde otro enfoque, esto implica que los beneficios promedios
pasaron de representar un 76\% a un 87\% del salario promedio del activo
en un periodo de 5 años.

\textbf{***agregar el gráfico Evolución anual del salario medio del activo sobre el beneficio medio del pasivo***}
