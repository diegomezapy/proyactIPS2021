\section{Metodología, Datos y Supuestos }\subsection{Metodología del Modelo de Proyecciones}

Para la realización del presente trabajo se utilizó como referencia la
metodología integral elaborada por el Servicio Financiero, Actuarial y
Estadístico de la OIT para el análisis de la situación actuarial y
financiera a largo plazo del Régimen de Pens iones. Se ha realizado la
modificación de la versión genérica de las herramientas de modelización
de la OIT, a efectos de adaptarlas a la situación del IPS, estas
herramientas de modelización incluyen el modelo demográfico, un modelo
de salarios y un mo delo de cobertura (OIT, 2002).

Los modelos de altas y evolución tanto de beneficios como beneficiarios
han sido desarrollados íntegramente por la Asesoría Actuarial del IPS en
función de la legislación vigente y la información disponible, de manera
a contar con un modelo de proyeccio nes que represente con mayor
precisión las características particulares del Sistema Jubilatorio del
IPS.

La valuación actuarial se inicia con una proyección del entorno
demográfico y económico del Paraguay, posteriormente se determinan y
utilizan las variables de proyección relacionadas con el Sistema de
Jubilaciones del IPS en combinación con el marco dem ográfico/económico.

Lo que se presenta en las siguientes secciones constituye una síntesis
de la metodología aplicada para obtener todas las proyecciones
actuariales.

\subsection{Fuente de datos}

Para realizar las proyecciones actuariales del Sistema de Jubilaciones y
Pensiones del IPS se requiere información específica para el régimen que
incluya indicadores demográficos, del mercado laboral, estadísticas
económicas y principalmente información acerca de las características de
los asegurados y las prestaciones que se pagan en el IPS.

Las fuentes utilizadas para la elaboración de los indicadores e insumos
necesarios son, en primer lugar, los resultados del último censo de
población, publicados por la DGEEC. En segundo lugar, se aprovechan los
microdatos de la EPH de los años 2008 al 2020, y para los datos
referentes a los indicadores económicos se utilizan las estadísticas
publicadas por el Banco Central del Paraguay. Por último, los insumos
más importantes son los que se obtienen de los Registros Administrativos
(RA) del IPS sobre los trabajadores activos, jubilados y pensionados
para el periodo 2008 al 2020. A continuación, se describe con mayor
detalle las informaciones recabadas de cada una de las fuentes citadas.

\subsubsection{El Censo Nacional de Población}

El Censo de Población, realizado cada 10 años por la DGEEC, representa
una fuente importante de información acerca de indicadores demográficos
relevantes para este informe. En la última publicación, basada en el
censo realizado en el año 2012, se dispon e de las proyecciones de
población nacional por sexo y edades individuales y quinquenales desde
el año 2000 al 2025 y una serie de indicadores demográficos, además de
los resultados de la proyección de población por sexo y grupos de edad,
según áreas ur bana y rural (DGEEC, 2016). Particularmente para este
estudio se utilizan los resultados relativos a la distribución de la
población por edad y sexo, la Tasa de Fertilidad por grupos
quinquenales, la Esperanza de Vida al nacimiento, la Tasa de Migración
Neta y la Tasa de Masculini dad. Estos son los insumos fundamentales
para el modelo de proyección de población.

\subsubsection{La Encuesta Permanente de Hogares }

Importantes indicadores sobre el mercado laboral que no se disponen en
los censos se pueden obtener de la EPH. Esta encuesta es ejecutada cada
año también por la DGEEC con el objeto de ofrecer información sobre los
principales indicadores de empleo del país, correspondiente a un periodo
específico, como una contribución para el análisis y la toma de
decisiones que tienda a la implementación de medidas para el
mejoramiento de las condiciones de vida de la población paraguaya
(DGEEC, 2016).

La EPH constituye la principal fuente de información sobre las
características socio-económicas del país. Se trata de una encuesta
representativa de toda la población, lo que permite realizar inferencias
poblacionales sobre una gran variedad de fenómeno s. Esta es su
principal ventaja sobre la encuesta de empleo y sobre los Registros
Administrativos.

De esta encuesta se obtienen indicadores como la distribución por sexo y
edad de la población económicamente activa, los ocupados y desocupados.
Los microdatos permiten además calcular las ``estadísticas de familia'',
que son indicadores relativos a la po blación en edad de trabajar,
procurando aproximarse a las características de la población asegurada
del IPS (cotizantes activos y jubilados y pensionados). Éstos incluyen
datos acerca de la distribución por sexo y edad simple de la proporción
de asegura dos casados, así como la edad promedio de los esposos/as, el
número promedio de hijos y sus respectivas edades medias.

\subsubsection{Banco Central del Paraguay (BCP)}

A partir de los informes periódicos que publica el BCP es posible
obtener datos sobre indicadores económicos, disponibles en las
publicaciones referentes a las cuentas nacionales. Estas cuentas
constituyen un instrumento completo para conocer y entender el estado de
la situación económica, así como su desenvolvimiento en el tiempo. Las
estadísticas son de enorme utilidad para la toma de decisiones, tanto
del sector público como del sector privado. En el sector público, para
adoptar decisiones de polít ica económica que orienten adecuadamente el
funcionamiento del sistema económico, así como para evaluar los
resultados de la gestión económica; en el sector privado, para definir
políticas empresariales coherentes con el entorno macroeconómico (BCP,
201 7).

Los datos que son obtenidos de las publicaciones del BCP y que se
utilizan como insumos para las proyecciones actuariales se citan a
continuación: evolución del Salario Mínimo Legal, evolución del Índice
de Precios al Consumidor y su variación interanua l, evolución del
Producto Interno Bruto, la tasa de interés de los depósitos en cuentas
de ahorros con plazos mayores a un año (CDA) y finalmente la evolución
de la participación de las remuneraciones en el PIB.

\subsubsection{Registros Administrativos del IPS}

Con seguridad la fuente más importante de datos son los Registros
Administrativos del IPS, éstas proveen los insumos fundamentales para
los cálculos relacionados a las proyecciones de la población cubierta
activa y la población en etapa pasiva o benefic iaria de alguna
prestación de vejez, invalidez o muerte. Seguidamente se describen
algunos detalles al respecto de la compilación, gestión y organización
de los Registros Administrativos que son utilizados como insumo en los
cálculos actuariales.

\begin{enumerate}
    \item[a)] \textbf{Base de datos de Asegurados Activos}
    
 La Dirección de Aporte Obrero Patronal (DAOP) dispone de una base de datos propia con los registros de todos los cotizantes, la cual cuenta con información valiosa que permite reconocerlos, a través de un identificador personal, que proporciona diverso
s datos sobre la empresa y sobre la historia de cotizaciones de los trabajadores. 

Algunos de los datos obtenidos del procesamiento de esta base de datos son: cantidad de cotizantes activos del Fondo de Jubilaciones y Pensiones por edad simple y sexo, salarios promedios, entre otros. Los registros fueron sometidos a procedimientos de 
recodificación, imputación y validación de todos los campos disponibles, puesto que se ha verificado la presencia de datos faltantes, inconsistentes y erróneos. 

Dependiendo de los indicadores referentes a los cotizantes activos que se presentan a lo largo de este informe, se consideran diferentes definiciones para identificar al trabajador cotizante activo. Por ejemplo, para calcular la cobertura del IPS con re
specto a la población económicamente activa, a la población total ocupada o con relación a la población obligada, es necesario recurrir además a los datos de la EPH, cuyo periodo de tiempo hace referencia al último trimestre del año. Para este caso los 
cotizantes activos se definen como aquellos que registran al menos un aporte durante el año. Por lo que como insumo para la proyección de la población cubierta se consideran como cotizantes activos a aquellos que registran al menos un aporte durante tod
o el año, siguiendo las recomendaciones del libro de la OIT sobre los modelos actuariales en Seguridad Social (Plamondon, y otros, 2002).  

 \item[b)] \textbf{Base de datos de Asegurados Jubilados y Pensionados}
         
La Dirección de Jubilaciones también cuenta con un registro de los asegurados en etapa pasiva (Jubilados y Pensionados). De las estadísticas obtenidas de los registros se pueden citar: cantidad total (stock) de beneficios, monto mensual promedio de las 
prestaciones de edad avanzada (Jubilación Ordinaria, Anticipada y Proporcional), invalidez, viudez y orfandad pagados en un mes y año determinado, entre otros. Para este informe se pudo armar la evolución del stock mes a mes para los años xx al xx.

Al mismo tiempo, estos registros permiten obtener información detallada sobre el stock de beneficios a una fecha de corte y sobre la cantidad total de nuevos pagos mes a mes (altas). Los nuevos beneficios considerados para los cálculos actuariales hacen
 referencia a los nuevos pagos otorgados en J y P correspondientes a los meses de enero a diciembre de cada año y en éstas no se incluyen los pagos únicos (indemnizaciones) ni los pagos por complementos al SML.

\end{enumerate}

\subsection{Proyecciones de Población}

A continuación, se describen los pasos realizados para utilizar el
método de las cohortes para las proyecciones de población.

Se parte de los datos sobre la población en el año t según sexo y edad,
la que procede de datos históricos, la tasa de sobrevivencia según sexo
y edad, la migración neta según sexo y edad, la tasa de fertilidad según
edad de la madre y la proporción de niños hombres respecto de los
nacimientos. A partir de estos datos se calcula:

\begin{enumerate}
\item Distribución de la población total del año base en generaciones según edad y sexo (cohortes).
\item Estimación de los nacimientos mediante las tasas de fertilidad por edad de la población femenina.
\item Estimación de la transición año por año de cada cohorte, teniendo en cuenta los fallecimientos (tablas de mortalidad por sexo y año de cohorte) y las migraciones (tasas netas de migración).
\end{enumerate}

Tal como puede apreciarse, el resultado final del modelo corresponde a
la Población en el año t+1 según sexo y edad para todos los años de la
proyección. El Gráfico xx ilustra el procedimiento de la proyección
demográfica.

\newpage

\begin{center}
Diagrama de flujo de la proyección demográfica
\end{center}\begin{figure}[!ht]
  \centering
  \hspace*{-80pt}
  \begin{tikzpicture}[node distance = 4cm, auto]
    % nodes
    \node [block] (tasasob) { Tasa de sobrevivencia};
    \node [block, below of = tasasob] (mig) { Migración neta según edad y sexo};
    \node [block, below of = mig] (tasafer) { Tasa de fertilidad según edad de la madre };
    \node [block, below of = tasafer] (prophom) { Proporción de niños hombres respecto de los nacimientos};
    \node [block, right of = tasasob] (pobt) { Población del año t según edad y sexo};
    \node [block, below of = pobt] (pobt+1) { Población del año t +1 según edad y sexo (edad $>$ 1)};
    \node [block, below of = pobt+1] (nuna) { Números de nacimientos};
    \node [block, below of = nuna] (nunase) { Números de nacimientos según sexo};
    \node [block,  below left=2.3cm and -0.3cm of prophom] (pob0t+1) { Población de 0 años de edad, en el año t+1 según sexo.};
    \node (pobt+1es) [block, right of=nuna, xshift=1cm] {Población del año t +1 según edad y sexo};

    % edges
         \path [line] (pobt) -- (pobt+1);
         \path [line] (pobt+1) -- (nuna);
         \path [line] (nuna) -- (nunase);
         \path [line] (nunase) -- (pob0t+1);
         \path [line] (pobt+1) -- (pobt+1es);
    \draw [line] (pob0t+1) -|  (pobt+1es);
    \path [line] (tasasob) -- (pobt+1);
    \path [line] (mig) -- (pobt+1);
    \path [line] (tasafer) -- (nuna);
    \path [line] (prophom) -- (nunase);
    \draw [line] (tasasob) -| (pob0t+1);
    \draw [line] (mig) -| (pob0t+1);
    \draw [line] (pobt+1es) |- (pobt+1);
    
  \end{tikzpicture}
\end{figure}

\subsection{Proyecciones de la Fuerza de Trabajo}

Tomando como punto de partida los resultados de la proyección
demográfica para cada año, se procede a la estimación de la Fuerza
Laboral y su clasificación entre ocupados y desocupados. El proceso
realizado puede resumirse en los siguientes pasos:

\begin{enumerate}
\item Estimación de la Población en Edad de Trabajar para cada año de la proyección (15 a 69 años), a partir de las proyecciones de población por edad y sexo obtenidas anteriormente.
\item Estimación de la Población Económicamente Activa (PEA), aplicando a las distintas cohortes de la PET las tasas de participación respectivas para cada edad y sexo. Para el caso de los hombres las tasas se mantienen constantes para todo el periodo d
e proyección, para las mujeres se asume un aumento interanual de xx\% para cada edad. Lo que equivale a un aumento total aproximado del xx\% en la tasa desde el inicio hasta el último año de proyección. Esta determinación considera en esencia la posibil
idad de que la participación femenina en el mercado de trabajo presente un aumento relativo en el futuro, aunque sea mínimamente. 
\item Clasificación de la PEA en Ocupada y Desocupada, estimada a partir de la PEA mediante la aplicación de Tasas de Desocupación. Estas tasas de desocupación por sexo y edad simple se basan en lo observado en el año 2020 y se mantienen sin variación a
 lo largo de los xx años de proyecciones.
\end{enumerate}

El principal producto del proceso corresponde al total de la población
ocupada en el año t; esto sin restar importancia a algunos productos
intermedios (aquellos que sirven de base para cálculos posteriores)
tales como la PEA en el año t (su distribució n por sexo y edad) así
como el número de desocupados.

Los insumos del modelo son:

\begin{enumerate}

\item La población por sexo y edad en un año respectivo;

\item una selección de los rangos de edad en los que se considera a una persona en edad de trabajar; por lo general se considera los 15 años y como edad final los 69 años, pero esto puede modificarse en base, por ejemplo, a disposiciones legales;

\item tasa de participación bruta por edad y sexo para el año base, y

\item tasa de desocupación para el año base.

\end{enumerate}

Estos dos últimos insumos serán supuestos basados en datos históricos.

\begin{figure}[!ht]
  \centering
  \hspace*{-105pt}
  \begin{tikzpicture}[node distance = 3cm, auto]
    % nodes
    \node (ppesat) [block, xshift=2cm] { Población proyectada por edad y sexo en el año t};
    \node [block, below left=4.5cm and 2cm of ppesat] (tpge) { Tasa de participación Global por edad };
    \node [block, below of=tpge] (tdat) { Tasa de Desocupación en el año t};
    \node [block, below of = ppesat] (pett) { Población en edad de trabajar en el año t};
    \node [block, below of = pett] (peat) { Población económicamente activa en el año t};
    \node [block, below of = peat] (pdt) { Población desocupada en el año t};
    \node (sept) [block, right of=pett, xshift=2cm] {Selección de edades en las que se puede trabajar};
    \node (pot) [block, right of=peat, xshift=2cm] {Población ocupada en el año t};
   
    % edges
         \path [line] (ppesat) -- (pett);
         \path [line] (pett) -- (peat);
         \path [line] (peat) -- (pdt);
         \path [line] (pett) -- (sept);
         \path [line] (peat) -- (pot);
         \path [line] (tpge) -- (peat);
         \path [line] (tdat) -- (pdt);

  \end{tikzpicture}
\end{figure}

\subsection{Proyecciones Económicas}

Al contar con las proyecciones socio-demográficas se procede a la
proyección económica, la cual brinda diversos productos finales e
intermedios que se convertirán a su vez en insumos para el modelo de
pensiones.

Los pasos realizados para obtener los resultados de la proyección
económica son: \renewcommand{\theenumi}{\arabic{enumi}}

\begin{enumerate}
\item Estimación de la productividad de la mano de obra empleada en el año proyectado.
\item Estimación de la producción real de la economía en dicho año.
\item Proyección del valor del deflactor del PIB.
\item Estimación del PIB nominal en base al PIB real y al deflactor del PIB.
\item Distribución del PIB entre factores productivos; el más importante de todos para el enfoque del modelo corresponde al factor trabajo.
\item Proyección del salario promedio teniendo en cuenta el ingreso total y la cantidad de empleados.
\item Proyección de la tasa de interés nominal.
\end{enumerate}

Los dos resultados finales del modelo corresponden al salario promedio
en determinado año y a la tasa nominal de interés.

Existen resultados intermedios que merecen atención y son utilizados en
los modelos actuariales, estos son el PIB real, el PIB nominal y el
ingreso total dirigido a la mano de obra.

Los insumos que alimentan el modelo son:

\renewcommand{\theenumi}{\roman{enumi}}

\begin{enumerate}
\item número de empleados; 
\item la productividad per cápita en el año inicial;
\item el deflactor del PIB en el año inicial (se considera el valor de la inflación.); 
\item incremento proyectado de la productividad;
\item inflación proyectada;
\item tasa de interés real proyectada, y
\item proyección de la proporción del ingreso dirigido a la mano de obra.
\end{enumerate}

Los puntos del iv) al vii) corresponden a supuestos basados en datos
históricos La figura ilustra el procedimiento de la proyección. \%aqui
va la figura

\begin{figure}[!ht]
  \centering
  \hspace*{-20pt}
  \begin{tikzpicture}[node distance = 2.5cm, auto]
    % nodes
   \node [block] (et) { Empleados en el año t};
   \node [block, below of = et ] (ppct) { Productividad per cápita, t}; 
   \node [block, below of = ppct ] (dpibt) { Deflactor del PIB en t}; 
   \node [block, below of = dpibt ] (pidmot) { Proporción del Ingreso dirigido a la Mano de Obra en t};
   \node [block, below of = pidmot ] (tint) { Tasa de interés nominal, en t};
   \node (pibrt) [block, right of=ppct, xshift=2cm] {PIB Real, t};
   \node (pibnt) [block, right of=dpibt, xshift=2cm] {PIB Nominal en t};
   \node [block,  below left=0.5cm and 2cm of ppct] (ip) { Incremento en productividad entre  t-1 y t};
   \node [block,  left of=ppct, xshift=-5cm] (ppct-1) { Productividad per cápita, t-1};
   \node [block,  left of=pidmot, xshift=-5cm] (cdpib) { Cambio en el deflactor del PIB };
   \node [block, below of = cdpib ] (it) { Inflación, t}; 
   \node [block, below of = it ] (tirt) { Tasa de interés real en t}; 
   \node (itdmot)  [block, right of=pidmot, xshift=2cm ] { Ingreso total dirigido a la mano de obra en t};
   \node (spt) [block, right of=tint, xshift=5cm] {Salario promedio en t};
 
   
    % edges
          \draw [line] (et) -| (spt);
          \draw [line] (itdmot) -| (spt);
          \path [line] (ppct) -- (pibrt);
          \path [line] (pibrt) -- (ppct);
          \path [line] (pibrt) -- (pibnt);
          \path [line] (dpibt) -- (pibnt);
          \path [line] (ppct-1) -- (ppct);
          \draw [line] (ip) |- (ppct);
          \draw [line] (cdpib) -- (dpibt);
          \path [line] (pibnt) -- (itdmot);
          \path [line] (pidmot) -- (itdmot);
          \path [line] (it) -- (cdpib);
          \path [line] (it) -- (tint);
          \draw [line] (tirt) -| (tint);

  \end{tikzpicture}
\end{figure}

\subsection{Proyecciones de Salarios}

Como insumo fundamental para estimar los ingresos por aportes
obrero-patronales, así como también para la determinación de los haberes
jubilatorios, se proyectan los promedios de los salarios de la economía
considerando tres niveles: altos, medios y baj os.

El proceso para la proyección de salarios puede resumirse de la
siguiente manera: \renewcommand{\theenumi}{\arabic{enumi}}

\begin{enumerate}
\item Estimación del salario promedio para un año base, por edad y sexo a partir de los R.A. (salarios declarados por los empleadores al Sistema de Aporte Obrero - Patronal).
\item Proyección del salario del año base, aplicando a los resultados anteriores la tasa de crecimiento de los salarios obtenida en el modelo de "Proyecciones Económicas".
\item Determinación de la desviación estándar, salario máximo y mínimo sujeto al Seguro Social.
\item Procesamiento del módulo de proyección según tipos de nivel salarial.
\end{enumerate}

El producto de este proceso es el salario promedio por grupos de
salarios (altos, medios y bajos) para todo el período de proyección por
edad simple y sexo. Los insumos de esta etapa son:
\renewcommand{\theenumi}{\roman{enumi}}

\begin{enumerate}
\item salarios sujetos a seguro que se obtienen de los R.A. del IPS;
\item población cotizante del Fondo de Jubilaciones por edad y sexo;
\item tasa de crecimiento de los salarios promedios;
\item desviación estándar de los salarios; y 
\item salarios declarados mínimos y máximos.
\end{enumerate}

\subsection{Proyecciones de la Cobertura}

Para la estimación de la población cubierta por el sistema en cada año
del periodo de proyección, los insumos del modelo son:

\begin{enumerate}
\item población ocupada y tasas de participación que se obtienen del modelo ILO-LAB;
\item tasa de cobertura inicial por edad simple y sexo;
\item tasa de incremento interanual global de la cobertura; ésta se mantiene constante durante todo el periodo de proyección y es igual a xx\%.
\end{enumerate}

Se estima la población cotizante para el periodo de proyección
2018-2100, para lo cual se considera la población cotizante en el año
2017, las probabilidades de muerte, invalidez, salidas y retiros para el
total del periodo de proyección.

El resultado corresponde a la población cotizante en el año t+1. En el
Gráfico xx se ilustra el procedimiento de la proyección.

\subsection{Proyecciones de Pensiones}

El modelo consta de archivos de insumos, de archivos de proyección, de
archivos de salida, archivos de base y de contabilidad a largo plazo.

Los datos de los archivos de insumos en su mayoría, se originan o
elaboran en los modelos anteriormente descritos (modelo de población, de
fuerza de trabajo, económico, de salarios, de población cubierta),
mientras que otros datos son directamente ingre sados en esta etapa.

El modelo genera como resultados una estimación para el período de
proyección de:

\begin{enumerate}
\item   Cuantía total de salarios sujetos a seguro y número de cotizantes.
\item   Cuantía total de los gastos en prestaciones y número de jubilados y pensionados.
\item   Situación del ingreso / gasto proyectado.
\end{enumerate}

Se asume como la edad de la viuda igual a 4 años menos que la edad del
jubilados fallecido y en caso de ser una mujer la jubilada fallecida, se
asume como la edad del viudo igual a 4 años más. Esta determinación es
obtenida de la diferencia media en eda des de los esposos hallado en la
EPH.

\begin{landscape}
Esquema del modelo de proyección para pensiones
\begin{figure}[!ht]
  \centering
  \hspace*{-25pt}
  \begin{tikzpicture}[node distance = 2cm, auto, inner sep=1pt]
    % nodes
\node [block] (pdna) {\scriptsize Proyección del número de aportantes};
\node (efd) [block, below left=1cm and -0.5cm of pdna] {\scriptsize Evolución futura de la densidad};
\node [block, below right=1cm and -1.3cm of efd] (phl){\scriptsize Proyección de la historia laboral};
\node [block, left of=efd, xshift=-2cm ] (da) {\scriptsize  Densidad actual};
\node [block,  below of = da ] (hldlpai) {\scriptsize Historia laboral pasada de la población asegurada inicial};
\node [block,  below of = hldlpai ] (cds) {\scriptsize Condiciones de selección};
\node [block,  below of = cds ] (cidjv) {\scriptsize Cantidad inicial de Jubilados por Vejez};
\node [block,  below of = cidjv ] (eppv) {\scriptsize Experiencia pasada en Prestaciones por Vejez};
\node [block,  below of = eppv ] (ciji) {\scriptsize Cantidad inicial de Jubilados por Invalidez};
\node [block,  below of = ciji ] (eppi) {\scriptsize Experiencia pasada en Prestaciones por Invalidez};
\node (tj) [block, right of=eppv, xshift=2cm] {\scriptsize Tasa de Jubilación};
\node (ti) [block, right of=eppi, xshift=2cm] {\scriptsize Tasa de Invalidez};
block2\node [block,  below of = phl ] (pspvb) {\scriptsize Población seleccionada para varios beneficios};
\node [block,  below of = pspvb ] (pnjpv) {\scriptsize Proyección del número de Jubilados por Vejez};
\node (pnji) [block, right of=ciji, xshift=5.5cm] {\scriptsize Proyección del número de Jubilados por Invalidez};
\node (pmtjv) [block, right of=pnjpv, xshift=2cm] {\scriptsize Proyección del monto total de Jubilaciones por Vejez};
\node (pmtji) [block, right of=pnji, xshift=2cm] {\scriptsize Proyección del monto total de Jubilaciones por Invalidez};
\node (pmpjv) [block, right of=pmtjv, xshift=2cm] {\scriptsize Proyección del monto promedio de Jubilaciones por Vejez};
\node (pmpji) [block, right of=pmtji, xshift=2cm] {\scriptsize Proyección del monto promedio de Jubilaciones por Invalidez};
\node (i) [block, right of=pmpjv, xshift=2cm] {\scriptsize Inflacion};
\node (i2) [block, right of=pmpji, xshift=2cm] {\scriptsize Inflacion};
\node (mppjv) [block, right of=pspvb, xshift=11cm] {\scriptsize Monto promedio de los pagos de Jubilaciones por Vejez};
\node [block, below right=0.5cm and -0.8cm of i] (fbjv){\scriptsize \scriptsize Fórmula del beneficio de Jubilación por Invalidez};
\node (mppji) [block, right of=i2, xshift=2cm] {\scriptsize Monto promedio de los pagos por Jubilaciones por Invalidez};
\node [block, below right=1.5cm and 3cm of i2] (fbji){\scriptsize \scriptsize Fórmula del beneficio de Jubilación por Invalidez};
\node (pipa) [block, right of=pdna, xshift=14cm] {\scriptsize Proyección de los ingresos promedios de los asegurados};



%edges
 \path [line] (pdna) -- (phl);
\path [line] (phl) -- (pspvb);
\path [line] (da) -- (efd);
\path [line] (efd) -- (phl);
\path [line] (hldlpai) -- (phl);
\path [line] (cds) -- (pspvb);
\path [line] (cidjv) -- (pnjpv);
\path [line] (eppv) -- (tj);
\path [line] (tj) -- (pnjpv);
\path [line] (ciji) -- (pnji);
\path [line] (eppi) -- (ti);
\path [line] (ti) -- (pnji);
\path [line] (pnjpv) -- (pmtjv);
\path [line] (pnji) -- (pmtji);
\path [line] (pmpjv) -- (pmtjv);
\path [line] (pmpji) -- (pmtji);
\path [line] (i) -- (pmpjv);
\path [line] (i2) -- (pmpji);

\path [line] (phl) -- (mppjv);
\path [line] (mppjv) -- (i);
\path [line] (fbjv) -- (i);
\path [line] (mppji) -- (i2);
\path [line] (fbji) -- (i2);

\draw [line] (pipa) -- (mppjv);
\draw [line] (pipa) -- (fbjv);
\draw [line] (pipa) -| (mppji);
\draw [line] (pipa) -| (fbji);
 
   \end{tikzpicture}
\end{figure}


\newpage
Esquema del modelo de proyección para pensiones(continuacion)
\begin{figure}[!ht]
  \centering
  \hspace*{-25pt}
  \begin{tikzpicture}[node distance = 2.5cm, auto]
    % nodes
\node [block] (cid) {\scriptsize Cantidad inicial de Derechohabientes};
\node [block, below right=1cm and -1.3cm of cid] (tm){\scriptsize Tasa de Mortalidad};
\node [block,  below of = tm ] (pecmm) {\scriptsize Probabilidad de estar casado al momento de la muerte};
\node [block,  below of = pecmm ] (ecamm) {\scriptsize Edad del cónyuge al momento de la muerte};
\node [block,  below of = ecamm ] (nhamm) {\scriptsize Número de hijos al momento de la muerte};
\node [block,  below of = nhamm ] (eh) {\scriptsize Edad de los hijos};
\node (dsm) [block, left of=pecmm, xshift=-3cm] {\scriptsize Datos sobre el matrimonio};
\node (dec) [block, left of=ecamm, xshift=-3cm] {\scriptsize Datos sobre la edad de los cónyuges};
\node (dsnh) [block, left of=nhamm, xshift=-3cm] {\scriptsize Datos sobre el número de hijos};
\node (dseh) [block, left of=eh, xshift=-3cm] {\scriptsize Datos sobre la edad de los hijos};
\node [block, below right=-0.5cm and 1cm of pecmm] (pnpdh){\scriptsize Proyección del número de Pensiones Derechohabiente};
\node (pmpdh) [block, right of=pnpdh, xshift=2cm] {\scriptsize Proyección del monto total de Pensiones Derechohabiente};
\node (pmppdh) [block, right of=pmpdh, xshift=2cm] {\scriptsize Proyección del monto promedio de Pensiones Derechohabiente};
\node (i) [block, right of=pmppdh, xshift=2cm] {\scriptsize Inflacion};
\node (mpppdh) [block, right of=tm, xshift=16cm] {\scriptsize Monto promedio de los pagos por Pensión Derechohabiente};
\node (fdbpdh) [block, right of=nhamm, xshift=16cm] {\scriptsize Fórmula del beneficio de Pensión Derechohabiente};
\node (ref) [, right of=cid, xshift=17.3cm] {};

%edges
\path [line] (dsm) -- (pecmm);
\path [line] (dec) -- (ecamm);
\path [line] (dsnh) -- (nhamm);
\path [line] (dseh) -- (eh);
\draw [line] (cid) -| (pnpdh);
\draw [line] (tm) -| (pnpdh);
\draw [line] (pecmm) -| (pnpdh);
\draw [line] (ecamm) -| (pnpdh);
\draw [line] (nhamm) -| (pnpdh);
\draw [line] (eh) -| (pnpdh);
\draw [line] (pnpdh) -- (pmpdh);
\draw [line] (pmppdh) -- (pmpdh);
\draw [line] (i) -- (pmppdh);

\draw [line] (mpppdh) -- (i);
\draw [line] (fdbpdh) -- (i);
\draw [line] (ref) -- (mpppdh);

 
   \end{tikzpicture}
\end{figure}
\end{landscape}

A continuación, se proveen más detalles de las fases principales que
componen las proyecciones de Jubilaciones y Pensiones.

\renewcommand{\theenumi}{\alph{enumi}} %Letras minúsculas

\begin{enumerate}
\item   \textbf{Prestaciones- Jubilación}

En esta fase se procede a realizar la proyección de trabajadores en etapa pasiva, la que comienza con el momento del retiro y ha sido parametrizado según los requisitos de edad y años de aporte según la legislación vigente.

La Ley 98/92 establece una "Jubilación Ordinaria" con una edad mínima de jubilación de 60 años y con 25 años de aportes y una Jubilación Anticipada con 55 años de edad y 30 o más años de aporte. Bajo este tipo de beneficio también se generan retiros ent
re los 56 y 59 años si el trabajador alcanza el citado requisito de años de aportes. A esta última prestación la denominaremos "Jubilación (Ordinaria) Anticipada" para diferenciarla de la "Jubilación Ordinaria".

La Ley Nº 4290/11 adicionalmente introduce la Jubilación Proporcional como uno de los beneficios jubilatorios a ser otorgados por el IPS. A la Jubilación Proporcional tendrá derecho el asegurado que se encuentre retirado de la actividad laboral, haya cu
mplido 65 (sesenta y cinco) años de edad y tenga 750 (setecientos cincuenta) semanas de cuotas como mínimo.

El último beneficio por vejez considerado concierne a las Leyes 430/74 y 375/56, y corresponde al rezago de las mismas. Se otorga a los que hasta el año 1999 habían cumplido los requisitos de edad y años de aportes.  Y si bien teóricamente no deberían e
xistir más altas de estos casos, en la práctica se siguen dando debido, tal vez, a un conocimiento tardío de dicho beneficio. Estas prestaciones no son contempladas en el módulo de proyecciones de pensiones, puesto que en la actualidad se observan muy p
ocos casos al año.

Una vez determinado el Haber Jubilatorio para el grupo de personas que se acoge al retiro en función al tipo de beneficio, el mismo se proyecta en función del ajuste general que debe ser equivalente a la variación del IPC publicado por el BCP.  Si bien 
dicho ajuste depende de ciertos requisitos, en el caso del IPS una Resolución del Consejo de Administración autoriza el aumento, previa recomendación de la oficina de Asesoría Actuarial. En la práctica se ha realizado todos los años y a efectos de esta 
proyección se aplicará el mismo criterio.

Si el Haber de la prestación es inferior al Haber Mínimo (HM), se abona un "complemento al mínimo", operación que se repite en cada período en que se presente esta situación. Esto es importante, si el HM sigue un patrón diferente al ajuste general puede
 generar que, casos que se encuentran por debajo del mínimo en un momento, luego puedan ser superiores al mismo.

Surge la dificultad de que el HM se paga sobre la suma de todas las prestaciones que percibe un jubilado o pensionado, que puede ser más de uno, en ese caso se paga como complemente la diferencia entre la suma de los haberes percibidos y el valor del HM
. En las proyecciones de pensiones se proyecta la cantidad de beneficios pagados y no la cantidad de beneficiarios, por lo que se omite la corrección de los montos al HM. Pero se verifica que el impacto en las erogaciones es mínimo. 

\textbf{Probabilidad de nuevos beneficios (altas) en Jubilación Ordinaria}

Para la cantidad de nuevos beneficios pagados por JO se obtienen primeramente las probabilidades en esta modalidad para cada sexo y edad particular.

Básicamente se calcula la probabilidad como el total de altas observadas para el sexo y la edad determinados de enero a diciembre del año 2017, dividido por la suma entre la cantidad de activos con al menos una cotización de enero a diciembre del año t=
2017 del mismo sexo, edad y los nuevos pagos observados de enero a diciembre del año 2017 del mismo sexo y edad. 

Considerando que "s" es el sexo del beneficiario, "x" es la edad, "t" el año, NBJO los nuevos beneficios pagados en JO y TA el total de cotizantes activos, la fórmula puede resumirse como: 

$$Prob. JO (s, x, t) = NBJO (s, x, t) / (TA (s, x, t) + NBJO (s, x, t))$$

\textbf{Probabilidad de nuevos beneficios (altas) en Jubilación Proporcional}

$$Prob. JP (s, x, t) = NBJP (s, x, t) / (TA (s, x, t) + NBJP (s, x, t))$$


\item\textbf{Prestaciones - Invalidez}
Para el otorgamiento de las Jubilaciones por Invalidez, se aplican las tasas de invalidez obtenidas por edad y sexo a los afiliados expuestos (cotizantes al Fondo de J y P menores a 61 años). Esta tasa se mantiene constante a lo largo del periodo de pro
yección y para obtener la cantidad de Jubilaciones de Invalidez se multiplican por la cantidad de aportantes proyectada. Para obtener los egresos por esta prestación se calcula el 60\% del salario promedio obtenido de las proyecciones de salarios. No se
 contempla la separación del beneficio en pagos periódicos (Jubilación Permanente) y pagos únicos (Jubilación Temporal).

\textbf{Probabilidad de nuevos beneficios (altas) en Jubilación de Invalidez}

Considerando que "s" es el sexo del beneficiario, "x" es la edad, "t" el año, NBJI los nuevos beneficios pagados en JI y TA el total de cotizantes activos, la fórmula puede resumirse como: 

$$Prob. JI (s, x, t) = NBI (s, x, t) / (TA (s, x, t) + NBI (s, x, t))$$

\item   \textbf{        Prestaciones - Pensiones}
En el caso de Pensiones, tanto las originadas del fallecimiento de un trabajador como derivadas de una Jubilación, se realiza una operación similar al módulo de Invalidez.  Inicialmente se aplican las tasas de mortalidad a los colectivos de afiliados ex
puestos (trabajadores activos), para luego aplicar los filtros por edad y años de aporte. Si se cumplen con los requisitos establecidos en la normativa, se otorgan beneficios en función de una matriz de vínculos familiares (estadísticas de familia), que
 indica la probabilidad de poseer cónyuge y la dispersión de sus edades. Estos valores fueron parametrizados en función a los datos obtenidos de los R.A. del IPS.

Existen dos casos particulares en los cuales se calcula el pago de un monto único en lugar de una renta vitalicia. El primero de estos corresponde a los casos en los que la viuda/o, concubina/o es menor de 40 años, en cuyos casos corresponde una indemni
zación equivalente a 3 anualidades de la pensión que le hubiese correspondido. El segundo caso se produce cuando el afiliado no cuenta con los 15 años de aportes mínimos, correspondiendo el pago de un salario mínimo mensual por cada año de aportes con q
ue cuente el titular. Ninguno de estos casos es considerado en las proyecciones de los gastos en J y P.

\item   \textbf{d)      Fondo de Reserva}
La cantidad de aportantes proyectada combinada con la estructura salarial por edad y sexo determina la masa salarial sobre la cual se aplica la alícuota correspondiente al aporte obrero-patronal del 12,5\%. Es importante mencionar que la masa salarial i
mponible corresponde a 12 salarios por año. 

Los Ingresos Totales se conforman con la recaudación del aporte Obrero-Patronal, el potencial Aporte Estatal del 1,5\% sobre la masa salarial, más la rentabilidad del Fondo de Reserva.  Atendiendo que en la práctica no se recibe el Aporte Estatal, se de
sarrollarán dos escenarios en las proyecciones, uno con y otro sin dicho aporte.

La diferencia entre la recaudación y los egresos por prestaciones determinan el "Superávit o Déficit Corriente", mientras que la diferencia entre los ingresos totales y los egresos en todo concepto determinan el "Superávit o Déficit Contable".  El Fondo
 de Reserva al final de un año es equivalente a dicho fondo al inicio del período, más el superávit corriente, más la rentabilidad obtenida durante el año.

\end{enumerate}
